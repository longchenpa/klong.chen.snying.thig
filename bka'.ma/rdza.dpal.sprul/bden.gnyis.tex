\documentclass{article}
\usepackage{hyperref}
\usepackage[english,ukrainian]{babel}
\usepackage[utf8]{inputenc}

\begin{document}

\title{Дві істини}
\author{Дза Патрул $^1$}
\date{ $^1$ Майстер лінії Лонгчен Нінгтік. }

\maketitle

\begin{abstract}Настанова щодо погляду Махаяни: Роз’яснення двох істин. \\
\textbf{Ключові слова}: Махаяна.
\end{abstract}

\tableofcontents

\vspace{1cm}

Для тих, хто прагне досягти звільнення, є два вчення: (1) вчення про те, що потрібно усвідомити, і (2) вчення про те, як це втілити на практиці.

\section{Вчення про те, що потрібно усвідомити}

У цьому розділі розглядаються два аспекти: (1.1) природний стан усіх явищ, що можна піз нати, загалом і (1.2) природний стан власного розуму.

\subsection{Природний стан усіх явищ, що можна пізнати}

Цей розділ також поділяється на два аспекти: (1.1.1) відносний і (1.1.2) абсолютний.

\subsubsection{Відносний аспект}
Загалом, усі явища — від найнижчого пекла Найвищого Страждання до післямедитаційного досвіду бодгісатв на десятому бхумі включно — є відносними. Більше того, є два види відносного: неправильне відносне і правильне відносне. Усе, що ми сприймаємо до того, як ступаємо на шлях, належить до категорії неправильного відносного. Коли ми досягаємо етапу «прагнення до поведінки»
\footnote{Тобто шляхи накопичення і з'єднання},якщо ми можемо інтегрувати певне усвідомлення у наш досвід, це стає правильним відносним, але коли ми цього не робимо, це залишається неправильним відносним. Після досягнення бхум усі явища, що постають перед

\subsubsection{Абсолютний аспект}

По суті, абсолют — це основний простір явищ (дгармадгату), вільний від усіх концептуальних ускладнень. У своїй сутності він не має поділів, але все ж можна говорити про «поділи» залежно від того, чи була ця реальність усвідомлена. Таким чином, є поділ на абсолют, який є самою основною природою, і абсолют, який є усвідомленням (або «проявленням») цієї основної природи. Також є поділ на абсолют, який прояснюється \footnote{Буквально «про які усуваються помилкові уявлення».} через вивчення і роздуми, і абсолют, який переживається через медитаційну практику; або абсолют, який концептуально виводиться звичайними істотами, проти абсолюту, який безпосередньо переживається шляхетними істотами. Існує також поділ на концептуальний абсолют (намдрангпе дьондам) і абсолют, що перебуває поза концептуалізацією (намдранг майінпе дьондам). Є три способи, якими ми можемо переживати ці дві істини:

\begin{itemize}
\item На етапі звичайних істот явища вважаються реальними за своєю природою і сприймаються з прив’язаністю. Це називається неправильним відносним.
\item На етапі шляхетних істот явища усвідомлюються як оманливі і сприймаються без прив’язаності. Це називається правильним відносним.
\item На етапі будди немає ні звичайних явищ, ні їх відсутності, і будь-які турботи про прив’язаність чи неприв’язаність більше не застосовуються. Це називається абсолютним.
\end{itemize}

Іншими словами, на першому етапі є і явища, і прив’язаність, на середньому етапі є лише явища без прив’язаності, а на останньому етапі немає ні явищ, ні прив’язаності. Ці три етапи також відомі як «неправильне знання», «знання розуміння відносного» і «знання розуміння абсолютного». У випадку звичайних істот мудрість розуміння відносного залежить від аналізу, але для шляхетних істот вона досягається через безпосереднє сприйняття. Хоча звичайні поняття, такі як «розуміння» чи «нерозуміння», не застосовуються до самого абсолютного простору реальності, ми все ще можемо використовувати терміни, такі як «розуміння» чи «усвідомлення», щоб позначити визнання цього стану. Зрештою, ми повинні усвідомити неподільність двох істин, але твердження, що відносне стосується існування, тоді як на абсолютному рівні речей не існує, ніколи не вважатиметься поглядом Серединного Шляху. Коли ми усвідомлюємо єдину справжню природу правильного відносного, дві істини зливаються нерозривно, поза концептуальними крайнощами існування, неіснування, сталості чи порожнечі. Як сказано в «Матері Праджняпараміти»:

Справжня природа відносного є справжньою природою абсолютного.

Поділ на дві істини є лише умовним прийомом, заснованим на різних перспективах двох станів розуму, що використовується для полегшення розуміння. Усі різноманітні сутності, які постають перед спантеличеним станом розуму, позначаються як «відносні», тоді як «абсолютне» стосується стану розуму, в якому спантеличеність припинилася і в якому немає навіть найменшого сліду концептуального фокусу, навіть щодо самого неіснування. Як сказано:

\begin{verse}
        Коли поняття реального і нереального \\
        Відсутні перед розумом, \\
        Немає іншої можливості, \\
        Окрім як перебувати в цілковитому спокої, поза концепціями. \footnote{Бодгічар’яаватара, IX, 34.} \\
\end{verse}

Насправді, у великому неконцептуальному просторі абсолютної сфери — остаточної природи, яку потрібно зрозуміти — немає підстави для розрізнення двох рівнів реальності, і тому такого поділу не існує. У кінцевому усвідомленні будди також немає поділу на два рівні істини. Навіть оманливі явища, які ми переживаємо прямо зараз, не складаються з двох окремих рівнів реальності; вони є просто нерозривною єдністю явищ і порожнечі, або усвідомлення і порожнечі.
Усвідомлення або повне розуміння цього є мудрістю розуму будди, недвозначною первинною свідомістю самої реальності. Розуміючи дві істини окремо, зрештою вони зливаються в нерозривну єдність, і це те, що ми називаємо «недвозначною мудрістю єдності» або «неперебуваючою нірваною» тощо.

\subsection{Природний стан власного розуму}

Хоча ми можемо зрозуміти природний стан усіх явищ, що можна пізнати,
описаним чином, якщо ми не зрозуміємо природний стан суб'єкта пізнання,
тобто нашого власного розуму, усі явища залишаться об'єктами знання,
і це не слугуватиме протиотрутою для наших ментальних страждань.
Насправді, це усвідомлення саме по собі стане причиною для почуття
гордині та зарозумілості, лише посилюючи наше відчуття особистого «я».
Ось чому ми повинні розпізнати справжню природу того,
хто має це усвідомлення — інтелекту, розуму чи свідомості.

Є два аспекти цього:
(1.2.1) тимчасове розуміння в термінах двох істин і
(1.2.2) остаточне розуміння, в якому істини є неподільними.

\subsubsection{Тимчасове розуміння в термінах двох істин}

Коли ми розуміємо природний стан явищ загалом і усвідомлюємо, що вони подібні до ілюзії на відносному рівні, оскільки з’являються, хоча не є по-справжньому існуючими, тоді як на абсолютному рівні вони подібні до простору, оскільки їхнє існування чи неіснування не може бути встановлено, а також розуміємо, що в кінцевому підсумку істини є неподільними в межах великого Серединного Шляху — абсолютного простору реальності поза всіма концептуальними крайнощами — розум або свідомість, що має це розуміння, є відносним. Як сказав Шантідева:

\begin{verse}
        Абсолют лежить поза межами інтелекту, \\
        Бо розум є лише відносним, як вчать. \footnote{Бодгічар’яаватара, IX, 2.}
\end{verse}

Інтелектуальний розум, що має таке розуміння, може стати гордовитим і зарозумілим. Така гординя і зарозумілість є агентами мари і лише псуватимуть наше розуміння. Як сказано в «Сутрі, що розкриває немислиму сферу будди»:

\begin{verse}
        Так зване «досягнення» є швидкоплинним, \\
        Так зване «усвідомлення» є зарозумілим припущенням. \\
        Швидкоплинне чи зарозуміле припущення має бути дією мари. \\
        Надзвичайно зарозумілі ті, хто думають: «Я цього досяг.» \\
        Або кажуть собі: «Я повністю зрозумів.»
\end{verse}

Природа відносного розуму, який розуміє, є абсолютною.
Якщо ми досліджуємо саму природу свідомості, розуму чи інтелекту,
що розуміє, ми не можемо знайти нічого реального чи сутнісного.
Насправді, вона завжди була вільною від існування і неіснування,
від виникнення і припинення, від приходу і відходу, від сталості
і порожнечі, від минулого, сьогодення чи майбутнього,
і тому вона є самою абсолютною реальністю.

«Сутра, запитана Кашьяпою», каже:

\begin{verse}
        Розум не можна знайти всередині. \\
        Він також не існує зовні. \\
        І його не можна спостерігати ніде інде.
\end{verse}

«Сутра, запитана Майтрейєю», каже:

\begin{verse}
        Розум не має форми, кольору чи місця. \\
        Він подібний до простору.
\end{verse}

\subsubsection{Отаточне розуміння, в якому істини є неподільними.}

У кінцевій природі розуму дві істини є неподільними. Застосування двох істин до єдиної природи розуму є лише умовним використанням позначень чи термінології. У базовому або абсолютному просторі реальності немає звичайного розуму, тому немає підстави, на якій можна було б застосовувати дві істини. Також немає звичайного розуму в плоді, мудрості розуму будди, тож і його не можна позначити в термінах двох істин. Навіть у ясності та порожнечі, що є природою розуму спантеличених живих істот, ми не можемо знайти цього [розрізнення], тому що є лише ясна свідомість і порожнеча. Ось чому ми повинні усвідомити, як дві істини є неподільними.
Проте, оскільки неподільність двох істин може бути усвідомлена лише після того, як ми зрозуміли характеристики кожної окремо, поділ на дві істини все ще має свою мету.

Таким чином, неконцептуальна простота, що є природним станом того, що потрібно пізнати, нерозривно зливається з неконцептуальною простотою природного стану розуму. У цьому досвіді, який вільний від будь-якого поняття про індивідуальне «я» чи феноменальну ідентичність, усі зовнішні та внутрішні явища сприймаються як ненароджений простір, вільний від будь-яких концептуальних конструктів, таких як існування, неіснування, сталість чи порожнеча тощо, і все ж цей досвід перебуває поза дуальністю того, що бачиться, і того, хто бачить, або того, що усвідомлюється, і того, хто усвідомлює. Тому це є досконалим, безпомилковим усвідомленням.

\section{Вчення про те, як це втілити на практиці}

Цей розділ має дві частини.

\subsection{Пряма практика для тих, що із найгострішими здібностямми}

Ті, хто в минулому накопичив дві акумуляції, і хто має глибоко добру карму та удачу, можуть досягти усвідомлення лише отримавши настанови про дві істини. У їхньому випадку достатньо просто підтримувати безперервність цього розпізнавання. У їхній медитаційній рівновазі, яка вільна від дуальності знання і того, що пізнається, і перебуває поза будь-яким поняттям про «я», вони медитуватимуть у простороподібний спосіб без будь-яких концептуальних ускладнень, пов’язаних із двома істинами. Під час такої медитації немає негативних думок, які потрібно усунути, і немає позитивних станів розуму, на яких потрібно зосереджуватися. Як сказав Будда Майтрея:

\begin{verse}
\item        У цьому немає нічого, що потрібно прибрати, \\
\item        Ані найменшої речі, яку потрібно додати. \\
\item        Це ідеальне споглядання самої реальності, \\
\item        І коли реальність побачили, настає повне звільнення. \footnote{Орнамент ясного усвідомлення, V, 21 і Високе продовження, I, 154. Це також вірш 7 із «Серця залежного виникнення» Нагарджуни.}
\end{verse}

Після цього, [у післямедитаційний період], людина підтримує подібний до сну
досвід єдності двох істин, розпізнаючи, як усе сприймане з’являється,
але не має справжньої реальності. Водночас, із ілюзорною бодгічіттою,
любов’ю і співчуттям до всіх ілюзорних, подібних до сну істот,
які цього не усвідомили, людина накопичує дві ілюзорні акумуляції
і робить великі молитви прагнення на їхню користь.

\subsection{Поступова практика для тих, що із менш гострими здібностями}

Ті, у кого здібності менш гострі, повинні тренуватися поступово, починаючи з чотирьох споглядань, що відвертають розум від самсари. Якщо вони не підуть цим шляхом, вони ніколи не вийдуть за межі концептуальних ідей про глибоке усвідомлення.

Сказано:

\begin{verse}
        Усі наші думки і сприйняття є відносними. \\
        Усвідомлення їхньої природи є абсолютним. \\
        Розум, який це усвідомлює, є відносним. \\
        Відсутність справжньої реальності розуму є абсолютним. \\
        Терміни, що позначають дві істини, є відносними. \\
        Відсутність справжньої реальності в таких термінах є абсолютним. \\
        Недвозначність цього є єдністю двох істин. \\
        У природі того, що пізнається, і в мудрості розуму будд \\
        Навіть єдність двох істин не може бути спостережена, \\
        І тому це називається «абсолютним простором поза ускладненнями». \\
        У ньому не можна знайти ні «я» індивіда, ні явищ. \\
        Усвідомлення цього є поглядом. \\
        Перебування в ньому є медитацією. \\
        Накопичення заради інших із співчуття є дією. \\
        Розчинення дуалістичного сприйняття в основному просторі є плодом. \\
        Мудрість, що пронизує все, представляє просвітлені якості. \\
        І природне здійснення блага інших є просвітленою діяльністю. \\
        Не прив’язуючись до слів і позначень, ніби вони самі є змістом, \\
        Скерувати розум замість цього до змісту, на який лише вказують слова.
\end{verse}

Справжній розум, який є суб’єктом досвіду явищ, є вільним від будь-якої справжньої реальності, і тому, стосовно цього, ми кажемо, що немає «я», немає живої істоти, немає індивіда, немає діяча тощо. Коли ми кажемо «ні» або «неіснуючий» у цьому контексті, це означає, що існування не може бути встановлено. Але оскільки існування не може бути встановлено, неіснування також не може бути встановлено, і тому термін «ні» позначає не-встановлення як існування, так і неіснування.

Ця свідомість, що сприймає свій об’єкт, не залежить від органів чуття. Вона не походить від об’єктів. І вона не перебуває десь посередині. Вона не існує ні всередині, ні зовні. Коли вона виникає, вона не приходить звідкись, і коли вона припиняється, вона не йде кудись. Вона є порожньою, коли виникає, і порожньою, коли припиняється. Саме так це описується. У сутрах, наприклад, ми знаходимо такі твердження:

У цьому досконалому баченні жодні явища не з’являться.

«Мати Праджняпараміта» каже:

\begin{verse}
        Концептуалізація — це залучення до сфери бажань, сфери форм або безформної сфери. \\
        Але неконцептуалізація не пов’язана з жодною з них. \\
\end{verse}

Сутра каже:

\begin{verse}
        Коли жодної діяльності взагалі не виконується, \\
        Це називається «йогічною дією».
\end{verse}

І:

\begin{verse}
        Тому підтримання звичайного стану, вільного від будь-яких дгарм, є найвищою Дгармою. \\
\end{verse}

Сутра каже:

\begin{verse}
        Що таке найвища Дгарма? \\
        Це відсутність будь-якого поняття про дгарми.
\end{verse}

«Мати Праджняпараміта» каже:

\begin{verse}
        Оскільки пробудження не може бути спостережене, «пробудження» — це лише назва. \\
        Оскільки буддійство не може бути спостережене, воно також є лише назвою. \\
        Усвідомлення того, що в простороподібному природному стані всіх явищ немає нічого, що могло б бути об’єктом свідомості чи мудрості, є поглядом. \\
        Перебування з цим розпізнаванням — у спосіб «не-перебування» — є медитацією. \\
        У післямедитаційний період накопичення ілюзорної заслуги заради ілюзорних живих істот є дією. \\
        Розчинення ілюзорних сприйнять розуму в основному просторі є остаточним плодом. \\
        Основний простір явищ є поза концептуальними ускладненнями і невиразний мовою чи думкою. \\
        У цьому немає знання якогось об’єкта, що потрібно пізнати. \\
        Але все ж кажуть, що є практика погляду і медитації, \\
        Як простір, що споглядає простір, або небо, що медитує над собою. \\
        У справжній реальності немає розуму і немає явищ, \\
        Але сказати «ні» означає, що навіть дихотомія існування і неіснування перевершується. \\
        Казано, що не боятися глибокого значення порожнечі, а надихатися ним — це ознака щасливої істоти, яка раніше слухала і тренувалася в навчаннях і якій судилося швидко досягти пробудження. \\
        Сама реальність, подібна до неба основного простору, вільна від будь-яких думок, \\
        Коли вона усвідомлюється в стані первинної мудрості поза вираженням, \\
        Є фундаментальною рівністю, вільною від спекуляцій чи навмисної діяльності. \\
        Це мудрість розуму будд трьох часів. \\
        Абсолют, природа самої реальності, подібна до дитини безплідної жінки, \\
        Ніщо не може проявитися чи з’явитися; \footnote{Видання si khron mi rigs dpe skrun khang має mi bsam. Видання Варанасі має mi gsal.} це просто стан найфундаментальнішої звичайності. \\
        Переживати зумовлені явища відносного, магічні явища єдності, \\
        Без прийняття чи відкидання їх і без прив’язаності, \\
        Це означає взяти мудрість розуму будд у досвід. \\
        Поки ви не досягнете цього рівня майстерності і досягнення розуму, \\
        Відмовтеся від будь-якої прив’язаності до матеріальних володінь, \\
        І тримайтеся ізольованих лісів і відлюдних місць, як дикий олень. \\
        Ось як залишатися на шляху, ніколи не відступаючи назад. \\
        Залишайтеся без радості чи смутку, прив’язаності чи відрази тощо \footnote{Видання Варанасі має sogs. Видання si khron має thogs.} \\
        До всіх обставин, зовнішніх і внутрішніх, сприятливих і несприятливих, \\
        І кожен досвід значно допоможе вам на вашому шляху. \\
        Ось як знайти стабільне усвідомлення ненародженої природи явищ. \\
        Коли мудрість усвідомлення подібної до неба природи розуму \\
        І співчуття до ілюзорних живих істот, яких не покидають, \\
        З’єднуються разом у супутньому погляді та діяльності, \\
        Велика неперебуваюча первинна мудрість буде швидко досягнута.
\end{verse}

«Сутра Нірвани» каже:

\begin{verse}
        Порожнеча означає сприйняття ні «порожнього», ні «непорожнього». \\
        Природне сяйво порожнечі може з’являтися як що завгодно. \\
        Оскільки воно є порожнім, коли з’являється, явища і порожнеча є єдністю. \\
        Це можна знати лише шляхом внутрішнього споглядання. \\
        Це в межах вашої власної самопізнаючої свідомості-мудрості.
\end{verse}

Мачік Лабдрьон сказала:

\begin{verse}
        Коли нічого взагалі не концептуалізується, \\
        Як можна заблукати? \\
        Знищте свої концепції. І відпочивайте.
\end{verse}

І:

\begin{verse}
        Оскільки розум не є дуальністю, \\
        Дивись, ніби немає нічого, на що потрібно дивитися. \\
        Цей наш розум не бачиться жодним «дивленням». \\
        Сама природа розуму не усвідомлюється шляхом «бачення». \\
        Насправді, немає найменшої частки \\
        Чогось, на що можна дивитися.
\end{verse}

Природа розуму, порожня і ясна та поза концептуальним фокусом,
є справжньою фундаментальною умовою. Оскільки ця чиста свідомість,
вільна від концептуальних конструктів і неможлива для визначення,
виникає безперервно \footnote{Різні тибетські видання кажуть ‘char sgom‘gags pa’i, але я читаю це як ‘char sgo ma‘gags pa’i, слідуючи роз’ясненню Рінгу Тулку Рінпоче.} як ілюзорні явища, що є її основним вираженням,
ми повинні повністю довіритися цьому стану поза прив’язаністю,
цьому стану, в якому немає розділення між медитацією і післямедитацією,
і в якому ясність і порожнеча є єдністю, і прийняти його до серця через практику.

\vspace{1cm}

\subsubsection*{Колофон}
\footnotesize
Написано Першим Дза Патрулом Рінпоче. Перекладено Адамом Пірсі, 2005, доповнено 2012 завдяки учням Рінгу Тулку Рінпоче.
Текст датований другою половиною 19 століття, написаний в монастирі Дзогчен, провінції Кхам, округу Деге, регіону Гардзе, в Тибеті.
Українською переклав Намдак Тонпа 13 серпня 2025 року в м. Київ.

%[1] Тобто шляхи накопичення і з’єднання.
%[2] Буквально «про які усуваються помилкові уявлення».
%[3] Бодгічар’яаватара, IX, 34.
%[4] Бодгічар’яаватара, IX, 2.
%[5] Орнамент ясного усвідомлення, V, 21 і Високе продовження, I, 154. Це також вірш 7 із «Серця залежного виникнення» Нагарджуни.
%[6] Видання si khron mi rigs dpe skrun khang має mi bsam. Видання Варанасі має mi gsal.
%[7] Видання Варанасі має sogs. Видання si khron має thogs.
%[8] Різні тибетські видання кажуть ‘char sgom‘gags pa’i, але я читаю це як ‘char sgo ma‘gags pa’i, слідуючи роз’ясненню Рінгу Тулку Рінпоче.

\end{document}
