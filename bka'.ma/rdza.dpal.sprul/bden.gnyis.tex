\documentclass{article}
\usepackage[english,ukrainian]{babel}
\usepackage[utf8]{inputenc}
\ProvidesPackage{dharma}
\usepackage{caption}
\usepackage{hyperref}
\usepackage{url}
\usepackage[utf8]{inputenc}
\setlength{\parindent}{15pt}
\newif\ifincludeTOC
\includeTOCtrue
\lefthyphenmin=1
\hyphenpenalty=100
\tolerance=11000
\emergencystretch=1em
\hfuzz=2pt
\vfuzz=2pt


\begin{document}

\title{Дві істини}
\author{Дза Патрул $^1$}
\date{ $^1$ Лонгчен Нінгтік \\ \today }

\maketitle

\begin{abstract}Настанова щодо погляду Махаяни: Роз’яснення двох істин. \\
\textbf{Keywords}: Махаякна
\end{abstract}

\ifincludeTOC
  \tableofcontents
\fi

\newpage

Для тих, хто прагне досягти звільнення, є два вчення: (I) вчення про те, що потрібно усвідомити, і (II) вчення про те, як це втілити на практиці.

\section{Вчення про те, що потрібно усвідомити}

У цьому розділі розглядаються два аспекти: (1) природний стан усіх явищ, що можна піз нати, загалом і (2) природний стан власного розуму.

\subsection{Природний стан усіх явищ, що можна пізнати}

Цей розділ також поділяється на два аспекти: (i) відносний і (ii) абсолютний.


\subsubsection{Відносний аспект}
Загалом, усі явища — від найнижчого пекла Найвищого Страждання до післямедитаційного досвіду бодгісатв на десятому бхумі включно — є відносними. Більше того, є два види відносного: неправильне відносне і правильне відносне. Усе, що ми сприймаємо до того, як ступаємо на шлях, належить до категорії неправильного відносного. Коли ми досягаємо етапу «прагнення до поведінки» [1], якщо ми можемо інтегрувати певне усвідомлення у наш досвід, це стає правильним відносним, але коли ми цього не робимо, це залишається неправильним відносним. Після досягнення бхум усі явища, що постають перед

\subsubsection{Абсолютний аспект}

\subsection{Природний стан власного розуму}

\subsubsection{Тимчасове розуміння в термінах двох істин}

\subsubsection{Отаточне розуміння, в якому істини є неподільними.}

\section{Вчення про те, як це втілити на практиці}

\subsection{Пряма практика для тих, що із найгострішими здібностямми}

\subsection{Поступова практика для тих, що із менш гострими здібностями}

\end{document}
