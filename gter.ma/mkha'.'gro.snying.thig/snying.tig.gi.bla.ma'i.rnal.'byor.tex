\documentclass{article}
\usepackage{indentfirst}
\usepackage{microtype}
\usepackage{enumitem}
\usepackage[utf8]{inputenc}
\usepackage[T2A]{fontenc}
\usepackage[ukrainian]{babel}
\usepackage{hyperref}

\begin{document}

\title{Гуру-йога Кхандро Нінгтік}
\author{Пема Ледрел Цал \thanks{ Терма землі, передана еманацією Падмасамбхави у вигляді монаха }}
\date{13 століття}

\maketitle

\begin{abstract}Цей текст представляє переклад і тлумачення практики Гуру-йоги Кхандро Нінгтік,
ключового елемента традиції Дзокчен у тибетському буддизмі. Переклад забезпечує автентичність
і відповідність оригінальному тибетському тексту. Документ включає візуалізації, молитви та
ритуальні інструкції, спрямовані на духовне просвітлення через медитацію та відданість Гуру. \\
\indent \textbf{Ключові слова}: Дзокчен, Гуру-йога, Кхандро Нінгтік, Падмасамбхава, тибетський буддизм.
\end{abstract}

\section*{Внесок автора}
Текст Гуру-йоги Кхандро Нінгтік приписується Пемі Ледрелу Цалу, який, за традицією, отримав його як терма землі від еманації Падмасамбхави. Переклад українською мовою виконаний Дорінг Лонгченом Осалом Ранджунгом, який також адаптував текст для сучасного використання, зберігши його духовну цілісність і автентичність. Оригінальний тибетський текст включено як додаток для порівняння та подальшого вивчення.

\tableofcontents

\vspace{1cm}

\section{Вступна частина}

Тут розпочинається практика Гуру Йоги Кхандро Нінгтік.
Я схиляюся перед Гуру, Ідамом та Дакіні.

\section{Візуалізація}

Я уявляю себе у формі Ідама, чистого від недоліків, на маківці якого,
на троні з коштовностей, сонця, місяця та лотоса, перебуває корінний
Гуру разом із Падмасамбхавою з Уддіяни. Вони мають природу великого
Ваджрадхари, кольору неба, тримають ваджру та дзвіночок, одягнені в
шовки, кістяні прикраси та коштовності. Їхня супутниця — Дакіні Єше
Цог’ял з дамару, кістяними оздобами та іншими прикрасами.
У правій руці вона тримає строкатий шарф наповнений нектаром,
а в лівій — шовковий шарф, яким вона його підтримує.

Праворуч від яб-юм перебуває Падма Ледре Цал і Ранджунг Дордже,
подібні до Джампаї Янга. У правій руці він тримає ніж, а в лівій — на
пелюстці лотоса — книгу Кхандро Нінгтік. Ліворуч — царський син Легден
та Шак’я Жону, подібні до Авалокітешвари, білого кольору, з одним
обличчям і чотирма руками. Дві верхні руки складені долонями разом
біля серця, нижня права рука перебирає чотки з Ятіг, а нижня ліва
тримає білий лотос.

Попереду — Ролпаї Ваджра та Дарма Ваджра, подібні до Ваджрапані,
синього кольору, з одним обличчям і двома руками, тримають ваджру
та дзвіночок. Позаду — Намкха Ваджра та Цултрім Бум, подібні до
Манджушрі, що розбурхує серце, у правій руці тримають посудину з
нектаром, а в лівій — чотки з рубінового лотоса, що оточені біля серця.

Над головою Гуру, на троні з сонця, місяця та лотоса, перебуває від'ядхара
Шрі Сімха, кольору людської плоті, з одним обличчям і двома руками,
у позі Вайрочани, одягнений у шовкові шати кольору павичевого пір’я
та обладунки пандіти, сидячи у ваджрній позі. Над його головою — відун
Гараб Дордже, білого кольору, з одним обличчям і двома руками, права
рука вказує в небо жестом підкорення демонів, ліва — у жесті надання
притулку, з пучком волосся на маківці та шістьма кістяними прикрасами,
сидячи у позі бодхісаттви. Над ним — великий Ваджрадхара, кольору неба,
тримає ваджру та дзвіночок, прикрашений шістьма кістяними оздобами.
Над ним — п’ять сімей Будд у формі яб-юм, із відповідними атрибутами
своїх сімей. На їхній маківці — Самантабхадра яб-юм, синього кольору,
з одним обличчям і двома руками, тримають ваджру та дзвіночок у позі
обіймів, прикрашені різноманітними оздобами, сидячи у ваджрній позі.

Вони оточені всіма Гуру лінії передачі, пов’язаними з ученням, усіма
Буддами та Бодхісаттвами, які благословляють і підтримують їх.

\subsection{Візуалізація складів}

На маківці всіх цих [божеств] уявляю склад ОМ, у горлі — АХ, у
серці — ХУМ. З ХУМ у серці випромінюються промені світла, що
викликають Будд і Бодхісаттв десяти напрямків, Гуру, Ідамів,
богів, Дакіні, захисників Дхарми, три основи та палаци. Я промовляю:
ВАДЖРА САМАДЖА, ДЗА ХУМ БАМ ХО, запрошуючи та розчиняючи їх у собі.

Потім я підношу семичастинну молитву та підношення зовнішнього,
внутрішнього та таємного аспектів. З вірою та відданістю я звертаюся з молитвою:

\subsection{Молитва-присвята}

ОМ АХ ХУМ \\

\noindent У палаці чистої дгармадхату, вільної від речовинності та меж, \\
Я схиляюся перед самопізнанням трьох часів, перед усіма Гуру. \\
У палаці чистої дгармадхату без об’єктів, \\
Я молюся до Самантабхадри яб-юм, щоб осягнути первинну чистоту. \\
У палаці сяючого самопізнання, спонтанно досконалого, \\
Я молюся до п’яти сімей яб-юм, щоб проявити тіло мудрості. \\
У палаці безмежного співчуття, \\
Я молюся до Ваджрадхари, щоб звільнитися від прив’язаності до явищ. \\
У палаці палаючого вогню йогічної практики, \\
Я молюся до Гараба Дордже, щоб зустрітися з істиною дгарми. \\
У палаці острова Сосалінг, \\
Я молюся до Шрі Сімхи, щоб осягнути нерозривне блаженство. \\
У палаці палаючого лотоса, \\
Я молюся до Падмасамбхави яб-юм, щоб осягнути ненароджену мудрість. \\
У палаці скелі Кхрамо, \\
Я молюся до Кхандро Нінгтік, щоб досягти повного розчинення тілесних складників. \\
У палаці двох чистих нагромаджень, \\
Я молюся до Падма Ледре Цала та Самовиниклого Ваджри, щоб здійснити благо для себе та інших. \\
У палаці неподільної генеративної та завершальної стадій, \\
Я молюся до Легдена та Шак’я Жону, щоб благо істот виникало без упереджень. \\
У палаці найвищого місця перебування Дхарми, \\
Я молюся до Г’юнтона Ролпаї Ваджри, щоб досягти стабільної реалізації. \\
У палаці, де явища проявляються як тіло Дхарми, \\
Я молюся до Дарма Ваджри та Намкха Ваджри, щоб визріти у безмежному навчанні. \\
У палаці звільнення від прив’язаності до ясності та порожнечі, \\
Я молюся до Цултріма Буму та Дхармалагші, щоб досвід і усвідомлення зростали. \\
У палаці центрального каналу в серці, \\
Я молюся до милостивого корінного Гуру, щоб спонтанно осягнути п’ять тіл. \\
У палаці чистого тіла, \\
Я молюся до мирних і гнівних Дакіні, щоб практика досягла своєї межі. \\
У палаці чистих заповітів, \\
Я молюся до захисників терма, щоб усунути несприятливі умови та перешкоди. \\
Я приймаю притулок у Гуру, Будді, коштовності. \\
Нехай я звільнюся від концепції “я” силою розуму! \\
Нехай бажання не виникають у свідомості! \\
Нехай я осягну нерождену природу розуму в єдиному моменті! \\
Нехай омани розчиняться самі собою! \\
Нехай явища проявляться як тіло Дхарми! \\

Я промовляю сто назв або стільки, скільки можу: \\

\noindent OM A HUM MAHĀ GURU SAMANTABHADRA YAB-YUM NAMO HUM \\
OM A HUM ПЯТЬ СІМЕЙ YAB-YUM NAMO HUM \\
OM A HUM GARAB DORJE NAMO HUM \\
OM A HUM SHRI SIMHA NAMO HUM \\
OM A HUM PEMA JUNNE NAMO HUM \\
OM A HUM YESHE TSOGYAL NAMO HUM \\
OM A HUM PADMA LADRÖL TSAL NAMO HUM \\
OM A HUM RANGJUNG DORJE NAMO HUM \\
OM A HUM GYALTSÉ LEKDENPA NAMO HUM \\
OM A HUM SHAKYA ZHÖNNU NAMO HUM \\
OM A HUM ROLPÉ DORJE NAMO HUM \\
OM A HUM DHARMA DORJE NAMO HUM \\
OM A HUM NAMKHA DORJE NAMO HUM \\
OM A HUM TSULTRIM BUMPA NAMO HUM \\
OM A HUM DHARMA LAKSHA NAMO HUM \\
OM A HUM SUSKYE LAKSHA NAMO HUM \\

\subsection{Завершення візуалізації}

Під час промовляння молитов із тіл Гуру лінії передачі струмує нектар, що
входить через мою маківку, очищаючи всі гріхи та затемнення трьох воріт
(тіла, мовлення, розуму). Він виходить через пори ніг і пальців, наповнюючи
моє тіло, зовнішнє та внутрішнє, нектаром мудрості.

Потім Гуру, Ідами, Дакіні та захисники Дхарми розчиняються в Падмасамбхаві
яб-юм. З трьох місць (маківка, горло, серце) Гуру яб-юм виходять сіллаби
ОМ, АХ, ХУМ разом із променями світла, що розчиняються в моїх трьох місцях,
очищаючи затемнення трьох воріт і даруючи досягнення тіла, мовлення та розуму.

Потім Гуру яб-юм спускається вниз і розчиняється в корінному Гуру на
восьмипелюстковому лотосі в моєму серці. У цей час я отримав із уст Гуру
знання про методи здійснення молитви-присвяти, утримання розуму,
практики дихання та підготовку до практики ХУМ. Нехай буде благо!

\section{Додаток 1. Тибетський текст оригінал}

\begin{verbatim}
@#/_snying tig gi bla ma'i rnal 'byor pa zhugs so/_!
/bla ma yi dam mkha' 'gro rnams la phyag 'tshal lo/
/rang nyid yi dam ltar ba skyon sa pa'i spyi bor/
/rin po che'i khri dang nyi zla pad+ma'i steng du:
/rtsa ba'i bla ma dang o rgyan pad+ma dbyings:
/med kyi ngo bo rdo rje 'chang chen po nam mkha'i ma dog can rdo rje
dang du'i la bya' rdzin pa/
/dar dang dus pa dang rin po che'i brgya pa/
/yum mkha' 'gro mtsho rgyal da ma ra mo rus pa' rgyan dang ba cas pa/
/phyag g.yas ya bal 'khra+tsad cid spyod pa bdud rtsis bkang bag yon
pas ya bal stob pa/
/yab yum gnyis kyi g.yas su pad+ma las 'brel rtsal wA_dang/
/rang byung rdo rje'ajam pa'i dbyangs dang 'dra ba/
/g.yas ral gri dang g.yon ut+pa la'i steng na snying tig gi po ti bsnams pa/
/g.yon na rgyal sras legs pa dang shAkya gzhon nu gnyis spyan ras
gzigs dang 'dra ba sku mdog dkar po zhal gcig phyag bzhi pa/
/dang po gnyis thugasakar thal mo sbyar ba/
/_g.yas 'og ma chu:
/yatig gi phreng ba 'dren cing g.yon 'og mas pad+ma dkar po 'dzin pa/
/mdun du rol ba'i rdo rje dang dar ma rdo rje gnyis phyag rdor dang 'dra
ba sku mdog mthing kha zhal gcig phyag gnyis rdo rje dang dril bu 'dzin pa/
/rgyab tu nam mkha' rdo rje dang tshul khrims 'bum gnyis thugs rjechen
po dong sprug dang 'dra ba g.yas bdud rtsi'i bum

@#/_/pa_g.yon pad+ma rA ga'i phreng ba thugasakar 'dren pa/
/gu ru'i spyi gtsug tu nyi zla pad+ma'i gdan gyi steng du rig 'dzin shrI
sing ha sku mdog mi sha kha zhal/
/kyi gcig phyag gnyis rnam snang gi phyag rgya can/
/dar dur smrig gi na bza' gsol ba paN+Di ta'i cha byad can
rdo rje'i skyil krung du bzhugs pa/
/de'i spyi gtsug tu rig zla'adzin dga' rab rdo rje sku mdog dkar
po zhal gcig phyag gnyis g.yas sdigs mdzub nam mkhar gdengs pa/
/g.yon skyabs sbyin gyi phyag rgya can/_dbu skra thor gtsug dang
rus pa'i rgyan drug gsol ba/
/zhabs sems dpa'i skyil krung du bzhugs pa/_de'i spyi gtsug tu
rdo rje'achang chen po nam mkha'i mdog can rdo rje dang dril/
/bu 'dzin cing rus pa'i rgyan drug gsol ba/
/de'i spyi bor rigs lnga yab yum phyag mtshan rigs dang mthun pa/
/de'i spyi gtsug tu chos sku kun bzang yab yum sku/
/mdog mthing kha zhal gcig phyag gnyis rdo rje dang dril bu 'dzin
pas 'khril sbyor rgyan sna tshogs kyis brgyan pa rdo rje'i skyil
krung gis bzhugs pa/
/de la chos 'brel wA_yod tshad kyi bla ma brgyud pa
rnams dang sangs rgyas dang byang chub sems dpa' thams
cad kyis byin gyis rlob cing 'dren pa'i tshul du bskor nas bzhugs pa/
de rnams kyi spyi bor oM/
mgrin par AH/
thugs kar hU~M bsam/
thugs ka'i hU~M las 'od zer 'phros phyogs bcu'i
sangs rgyas dang byang chub sems dpa' bla ma yi:
ga_dam lha tshogs mkha' 'gro chos skyong rten gsum
pho brang dang bcas pa byon par bsam/
/badz+ra sa mA dzadang/
/dzahU~M baM ho:
/brjod pas spyan drang zhing bstim par bya'o/
/de la yan lag bdun pa dang phyi nang gsang gsum gyi mchod pa 'bul/
/de nas mos gus dang ldan pas gsol ba 'debs pa ni/
oM AhU~M hU~M:
dngos lo

@#/_/med mtha' bral ka dag chos kyi sku/
/rang rig gsal dwangs lhun grub longs spyod rdzogs/
/kun khyab shar grol snatshogs sprul pa'i sku/
/dus glu ka'i gsum rang rig bla ma rnams la 'dud/
/dmigs med chos dbyings dag pa'i pho brang du/
/chos sku kun bzang yab yum la gsol ba 'debs/
/ngo bo ka nas brgya dag par byin gyis rlobs/
/lhun grub rang gsal 'od kyi pho brang du/
/longs sku rigs lnga yab yum la gsol ba 'debs/
/rig pa sku ru 'char bar byin gyis zhe
bzhi rlobs thug rje phyogs ris med pa'i pho brang du/
/sprul sku rdo rje 'chang la gsol ba 'debs/
/snang sems 'dzin pa stongs par byin gyis rlobs/
/dur khrod rnal 'byor me ri 'bar ba'i pho brang du/
/rig 'dzin dga' rab Do rje la gsol ba 'debs/
/chos nyid don dang mjal bar byin gyis rlobs/
/rgya gar so sa gling gi dur khrod du rig 'dzin shrI sing hal gsol ba 'debs/
/bde chen 'du 'bral med par byin gyis rlobs/
/dpal ri pad+ma 'od kyi pho brang du/
/o rgyan pad+ma yab yum wA_la gsol ba 'debs/
/lhan skyes ye shes rtogs par byin gyis rlobs/
/gter gnas khra mo brag gi pho brang du/
/mkha' 'gro snying tig nyid la gsol ba 'debs/
/phung po lhag med 'grub par byin gyis rlobs/
/tshogs gnyis rnam par dag pa'i pho brang du/
/las 'brel rtsal dang rang byung rdo rje la gsol ba 'debs/
/bdag gzhan don gnyisa'agrub par byin gyis rlobs/
/bskyed rdzogs zung du 'jug pa'i pho brang du/
/legs ldan pa dang shAkya gzhon zhu la gsol ba'adebs/
/'gro don phyogs med 'byung bar byin gyis rlobs/
/gar bzhugs gnas mchog chos kyi pho brang du/
/rgyal ba g.yung ston rol pa'i rdo rje la gsol

@#/_/ba 'debs/
/bskyed rdzogs brtan pa thob par byin gyis rlobs/
//snang srid chos skur shar ba'i pho brang du/
/dar ma badz+ranam mkha' rdo rjel gsol ba 'debs/
/_phyogs med gdul byar smin par byin gyis rlobs/
/gsal stong 'dzin pa bral ba'i pho brang du/
/tshul khrims 'bum dang d+harma lak+Sha la gsol ba 'debs/
/nyams rtogs gong du 'phel bar byin gyis rlobs/
/snying dbus ma ba d+hA ti'i pho brang du/
/drin can rtsa ba'i bla ma la gsol ba 'debs/
/sku lnga lhun gyis 'grub par byin gyis rlobs/
/rang lus rnam par dag pa'i pho brang du/
/zhi khro DAk+ki'i tshogs la gsol ba 'debs/
/nyams len tshad du skyol bar byin gyis rlobs/
/dmatshig rnam par dag pa'i pho brang du/
/bka' srung gter bdag rnams la gsol ba 'debs/
/rkyen ngan bar chad med par byin gyis rlobs/
/bla ma sangs rgyas rin po che la skyabs su mchi/
/bdag 'dzin blo yis thongs par byin gyis rlobs/
/dgos med rgyud la skye bar byin gyis rlobs/
/rang wA_sems skye med cig char rtogs par byin gyis rlobs/
/'khrul pa rang sar dag par byin gyis rlobs/
/snang srid chos skur 'char bar byin gyis rlobs/
/mtshan 'bum ci gang 'don/
!oM AHhU~M ma hA gu ru kun bzang yab yum na mo hU~M/
/oM AHhU~M rigs lnga yab yum na mo hU~M/
/oM A:hU~M dga' rab rdo rje na mo hU~M/
/ga_mo hU~M shrI sing ha na mo hU~M/
/oM AHhU~M pad+ma'abyung gnas zha mo cu/
/oM AhU~M ye shes 'tsho rgyal na mo hU~M/
/oM A:hU~M pad+ma las 'brel rtsal na mo hU~M/
/oM AHhU~M rang byung rdo rje na mocu/
/oM AhU~M rgyal sras legs pa na mo hU~M/
/o~M A:hU~M shAkya gzhon nu na lo hU~M/
/oM AHhU~M rol pa'i rdo rje na mo hU~M/
/oM AHhU~M dar ma rdo rjen

@#/_/mohu/
/oM AhUM ga ga badz+ran mo hU~M/
/oM AHhU~M tshul khrims 'bum na mo hU~M/
/oM AHhUM d+harma lak+Sha na mohUM/
/oM AHhU~M su_skyelak+Shan mo hU~M//

/zhes gsol ba 'debs pa'i dus su bka' brgyud bla ma
rnams kyi sku las bdud rtsi byung nas rang gi spyi bo nas babs/
sgo gsum gyi sdig sgrib thams cad dag_/
rkang mthil dang sen mo'i sbug nas mar thon nas lus phyi
nang thams cad ye shes kyi bdud rtsis gang bar bsam mo/
/de nas bla ma yi dam mkha' 'gro chos skyong srung ma
rnams o rgyan pad+ma yab yumal bstim/
bla ma yab yum gyi gnas gsum nas oM AHhU~M
gsum 'od zer dang bcas pa byung nas rang gi gnas gsum du thim/
sgo rgyu_gsum gyi sgrib pa dag sku gsung thugs kyi
dngos grub thob par bsam/
de nas bla ma yab yum mar byon nas snying
pad+ma 'dab brgyad kyi steng du rtsa ba'i bla ma bstim mo/
/skabs 'dir gsol ba gdab pa dang/
/sems 'dzin byed lugs dang/
/rlung gi thun dang/
/hU~M sgrub sbyor ba rnams bla ma'i zhal las shes so// //dge'o/
\end{verbatim}

\end{document}
