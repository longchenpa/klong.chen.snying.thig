\documentclass{article}

\usepackage[utf8]{inputenc}
\usepackage[T2A]{fontenc}
\usepackage[ukrainian]{babel}

% Title and author
\title{Порівняльний аналіз каталогів Кхандро Нінгтік у повній антології Лонгчен Рабджама та Деге версії Нінгік Ябжі Тартанг Тулку}
\author{Максим Сохацький $^1$}
\date{ $^1$ Інститут тибетської літератури \\ \today }


\begin{document}

\maketitle

\begin{abstract}
Ця стаття порівнює каталоги (dkar chag) циклу Кхандро Нінгтік (mkha' 'gro snying thig),
частини Нінгтік Ябжі (snying thig ya bzhi) у традиції Ньїнґма. Аналіз зосереджено на двох
джерелах: повній антології Лонгчен Рабджама (sung bum dri med 'od zer), яка містить 26 томів
та виданні Нінгтік Ябжі Деге від Тартанг Тулку (ye shes sde), яке містить 5 томів.
Досліджується структура, зміст, текстові відмінності та контекст, що підкреслюють
начення для текстологічних досліджень терма циклів Дзогчену (rdzogs chen).

Ключові слова: Кхандро Нінгтік (mkha' 'gro snying thig), Лонгчен Рабджам (klong chen rab 'byams),
  Нінгтік Ябжі (snying thig ya bzhi), Дзогчен (rdzogs chen), каталог (dkar chag).
\end{abstract}

\newpage

\section{Вступ}

Кхандро Ньїнґтік є основним компонентом Ньїнґтіґ Ябджі, збірки вчень Дзоґчен,
що приписуються Лонгчен Рабджаму та іншим носіям лінії. Змісти Кхандро
Ньїнґтік з двох джерел — Сунгбум Дріме Озера (Лонгчена Рабджама) та видання
Деге Тартанг Тулку (Єше Де) — надають структурований огляд текстів цього циклу.
Це порівняння аналізує структурну організацію, текстовий зміст та відмінності між
двома покажчиками, підкреслюючи їх обсяг, розташування та значення для текстуальних досліджень.

\section{Методологія}
Аналіз базується на порівнянні документів:
\begin{itemize}
    \item Збірка Лонгчен Рабджама (sung bum dri med 'od zer): Каталог Кхандро Нінгтік у п'ятому та шостому томах.
    \item Видання Деге від Тартанг Тулку (ye shes sde): Каталог у третьому та п'ятому томах.
\end{itemize}

Порівняння оцінює:

\begin{itemize}
\item Структурну організацію: Розташування розділів, томів та текстів.
\item Охоплення змісту: Перелічені тексти, їх назви та тематична спрямованість.
\item Текстові відмінності: Варіації в назвах, пропуски, доповнення або порядок.
\item Метадані та контекст: Додаткова інформація, така як вступні зауваження, біографії або історії ліній походження.
\end{itemize}

\section{Аналіз}

\subsection{Збірка Лонгчен Рабджама}

\subsubsection{Історії}
\subsubsection{Тантри}
\subsubsection{Ритуали}
\subsubsection{Настанови}
\subsubsection{Доповнення}

\subsection{Видання Деге}

\subsubsection{Історії}

\begin{itemize}
\item mkha' 'gro snying thig gi kha byang --- огляд ліній та пророцтво.
\item rdzogs pa chen po mkha' 'gro snying thig gi bla ma brgyud pa'i lo rgyus --- історія лінії вчителів Великої Досконалості Кхандро Нінгтік.
\item mkha' 'gro snying thig gi lo rgyus --- загальна історія циклу.
\item sangs rgyas kyi 'das rjes dang po --- Перший заповіт Будди.
\item sangs rgyas kyi 'das rjes gnyis pa --- Другий заповіт Будди.
\item sangs rgyas kyi 'das rjes gsum pa --- Третій заповіт Будди.
\end{itemize}

\subsubsection{Тантри}
\subsubsection{Ритуали}
\subsubsection{Настанови}
\subsubsection{Доповнення}

\section{Текстові відмінності}
\begin{itemize}
    \item \textbf{Спільні тексти}: Шість тантр (btags grol gyi rgyud drug), тіки, настанови (thod rgal gyi don khrid nor bu'i snying po).
    \item \textbf{Унікальне в Sungbum}: \textit{dbang gong ma gsang dbang shes dbang tshig dbang gsum gyi sa ma 'grel}, стислі ритуали.
    \item \textbf{Унікальне в Деге}: \textit{g.yung ston pas gsungs pa'i phyag 'tshal}, \textit{gter srung ldang lha'i sgrub thabs}, \textit{kha skong} (наприклад, yang gsang mkha' 'gro snying tig gi khrid yig).
    \item \textbf{Назви}: Варіації, наприклад, \textit{snying po gsal ba'i me long zhes bya'i khrid} (Sungbum) проти \textit{snying thig gsang ba'i yang bcud snying po gsal ba'i me long} (Деге).
    \item \textbf{Порядок}: Sungbum компактніший, Деге складніший із \textit{e pa}, \textit{waM pa}, \textit{kha skong}.
\end{itemize}

\section{Метадані та контекст}
\begin{itemize}
    \item \textbf{Sungbum}: Включає зображення вчителя (ston pa 'jam dpal 'jigs skyob kyi snang brnyan), каталог (dri med 'od zer gyi gsung 'bum deb lnga pa), підкреслює авторство Лонгчен Рабджама.
    \item \textbf{Деге}: Розширені історії (lo rgyus rin po che'i phreng ba), без вступних зауважень, але з ширшим контекстом, можливо, від пізніших утримувачів.
\end{itemize}

\textbf{Порівняння}: Sungbum пов’язаний із Лонгчен Рабджамом, Деге відображає ширший редакційний обсяг.

\section{Обговорення}
\begin{itemize}
    \item \textbf{Sungbum}: Стислий, авторитетний, відображає систематизацію Лонгчен Рабджама.
    \item \textbf{Деге}: Розширений, із додатковими текстами, можливо, від регіональних традицій.
    \item \textbf{Наслідки}: Sungbum для дослідників Лонгчен Рабджама, Деге для практиків із ширшим спектром ритуалів.
\end{itemize}

\section{Висновки}
Каталоги Кхандро Нінгтік мають спільну основу, але різняться обсягом і організацією. Sungbum компактний, Деге розширений, що відображає різні редакційні підходи та підкреслює динаміку традицій Ньїнґма.

\end{document}