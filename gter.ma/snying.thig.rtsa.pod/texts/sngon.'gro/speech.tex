\small
\ti
༄༅། གདོད་ནས་མངནོ་སུམ་སངས་རྒྱས་ཀྱང་གང་ལ་གང་འདུལ་གཟུགས་སྐུ་འགགས་པ་མེད།\\
སྣ་ཚོགས་སྒྱུ་འཕྲུལ་ངོམ་ཡང་ཕུང་ཁམས་སྐྱེ་མཆེད་གཟུགས་དང་འཛིན་པ་བྲལ།\\
མི་ཡི་གཟུགས་སུ་སྣང་ཡང་མཁྱེན་བརྩེའི་འོད་ཟེར་སྟོང་འབར་རྒྱལ་བ་དངོས།\\
ཚེ་འདི་ཙམ་དུ་མ་ཡིན་གཏན་གྱི་སྐྱབས་སུ་ཁྱེད་བསྟེན་བྱིན་གྱིས་རློབས།\\
\\
\\
\ru

Просвітлений із самого початку,\\
Не зупиняєшся приймати форми,\\
Щоб приручити кожного у власний спосіб.\\
Хоч тебе бачать як феномени, що виникли від дива,\\
Насправді ти є вільний до останньої краплини\\
Від скандх та почуттів, від елементів й ланцюгів чіплянь.\\
Хоч ти з’явився у людському тілі,\\
Насправді ти Будда, і сяєш ти на ділі\\
Безліччю променів всевідання і щастя.\\
Не тільки в цім житті, але й назавше\\
Приймаю в тобі пристань, святе Кх'єнце Озер.\\
Рясним благословенням сповни мене тепер!\\

\\
\\
\vspace{0.5cm}
\\
\scriptsize
\ti དེ་ལ་འདིར་རྫོགས་པ་ཆེན་པོ་ཀློང་ཆེན་སྙིང་ཐིག་གི་སྔོན་དུ་འགྲོ་བའི་ངག་འདོན་ཁྲིགས་སུ་བསྡེབས་པ་ལ།\\
\\
\ru Тут представлена практика Лонгчен Нінгтік Ньондро.\\

\normalsize
\newpage
\section*{Благословіння мовлення}
\\
\ti
ཨོཾ་ཨཱཿཧཱུྂ།\\
ལྕེ་དབང་རྂ་ཡིག་ལས་བྱུང་མེས་བསྲེགས་ནས། \\
འོད་དམར་རྣམ་པའི་རྡོ་རྗེ་རྩེ་གསུམ་སྦུབས།\\
ཨཱ་ལི་ཀཱ་ལིའི་མཐའ་སྐོར་རྟེན་འབྲེལ་སྙིང་།\\
མུ་ཏིག་ཕྲེང་བ་ལྟ་བུའི་ཡིག་འབྲུ་ལས།\\
འོད་འཕྲོས་རྒྱལ་བ་སྲས་བཅས་མཆོད་པས་མཉེས།\\
སླར་འདུས་ངག་སྒྲིབ་དག་ནས་གསུང་རྡོ་རྗེའི།\\
བྱིན་རླབས་དངོས་གྲུབ་ཐམས་ཅད་ཐོབ་པར་འགྱུར།\\
\\
\ru
OM A ХУМ \\

Вогонь, народжений зі складу РАМ, випалює язик, \\
І виникає трьохкінечна ваджра як червоний лик. \\
Санскритські голосні і приголосні, мантрою окутані,\\
Взаємозв'язного Походження --- Сутності здобутої.  \\
Ось зі складів, подібних до перлового намиста, \\
Виходить світло радості, мов райдуга барвиста, \\
Що Буддам здійснює приношення і Бодхісаттвам,\\
Й завжди вертається назад як вірне клятвам, \\
І тьмяність очища, і здобува оздоблення \\
Благословіння й сидхи всі ваджрного мовлення.\\

\newpage
\subsection*{Мантра голосних}
\\
\ti
ཨ་ཨཱ། ཨི་ཨཱི། ཨུ་ཨཱུ། རྀ་རཱྀ། ལྀ་ལཱྀ། ཨེ་ཨཻ། ཨོ་ཨཽ། ཨཾ་ཨཿ\\
\\
\ru
А аа І іі У уу Рі ріі Лі ліі Е ее О оо Ам А \hspace{1cm} 7 разів\\
\\
\subsection*{Мантра приголосних}
\\
\ti
ཀ་ཁ་ག་གྷ་ང་། \\
ཙ་ཚ་ཛ་ཛྷ་ཉ། \\
ཊ་ཋ་ཌ་ཌྷ་ཎ། \\
ཏ་ཐ་ད་དྷ་ན། \\
པ་ཕ་བ་བྷ་མ། \\
ཡ་ར་ལ་ཝ། \\
ཤ་ཥ་ས་ཧ་ཀྵཿ \\
\\
\ru
КА  КХА  ГА ГХА НА\\
ЦА  ЦХА  ДЗА ДЗХА НЬЯ\\
ТРА ТХРА ДРА ДХРА НРА\\
ТА  ТХА  ДА ДХА НА\\
ПА  ПХА  БА БХА МА\\
Я   РА   ЛА ВА\\
ЩА  КХА  СА ХА КЩА \hspace{1cm} 7 разів\\
\newpage
\subsection*{Мантра взаємозалежного виникнення}
\\
\ti
ཨོཾ་ཡེ་དྷརྨཱ་ཧེ་ཏུ་པྲ་བྷཱ་ཝཱ་ཧེ་ཏུནྟེ་ཥཱནྟ་ཐཱ་ག་ཏོ་ཧྱ་ཝ་དཏ།\\
 ཏེ་ཥཱཉྩ་ཡོ་ནི་རོ་དྷ་ཨེ་ཝྃ་ཝཱ་དཱི་མ་ཧཱ་ཤྲ་མ་ཎཿསྭཱ་ཧཱ།\\
 \\
\ru
ОМ ЙЕДХАРМА ХЕТУПРАБХАВА\\
ХЕТУНТЕКХАН ТАТХАГАТО\\
ХАЙОВАДАТ ТЕКХАНЦАЙО НІРОДХА\\
ЕВАМВАДІ МАХАШРАМАНА! СОХА \hspace{1cm} 7 разів\\

\subsection*{Мантри підсилення заслуг}
\\
\ti
ཨོཾ་སམྦྷ་ར་སམྦྷ་ར་བི་མ་ནཱ་སཱ་ར་མ་ཧཱ་ཛམྦྷ་ཧཱུྂ།\\
ཨོཾ་སྨ་ར་སྨ་ར་བི་མཱ་ན་སྐ་ར་མ་ཧཱ་ཛ་བ་ཧཱུྂ།\\
\\
\ru
ОМ САМБХАРА САМБХАРА\\
ВИМАНА САРА МАХА ДЗАМБХА ХУМ\\
ОМ МАРА МАРА ВІМАНА\\
КАРА МАХА ДЗАБА ХУМ \hspace{1cm} 7 разів\\
\\
\ti
ཨོཾ་རུ་ཙི་ར་མ་ཎི་པྲ་བརྡྷ་ནཱ་ཡེ་སྭཱ་ཧཱ།\\
ཧྲཱིཿབཛྲ་ཛི་ཧཱ་མནྟྲ་དྷ་ར་བརྡྷ་ནི་ཨོཾ།\\
\\
\ru
ОМ РУЦІ РАМАНІ ПРАВАРДАЄ СОХА\\
ХРИ БАДЗРА ДЗІХА\\
МАНТРА ДХАРА ВАР ДХАНІ ОМ \hspace{1cm} 3 рази\\
\\
\scriptsize
\ti ཡི་དམ་གྱི་བཟླས་པའི་ཐོག་མར་འདི་དང་དབྱངས་གསལ་རྟེན་སྙིང་བཅས་བཟླས་པས་ཕྲེང་བ་བྱིན་གྱིས་བརླབས་པ་དང་།\\
སྐབས་སུ་ཁ་ཟས་ལའང་བཏབ་གྲུབ་ན་ནུས་པའི་མཐུ་སྐྱེད་པར་གསུངས་སོ།\\
\\
\ru Для благословіння мали перед початком рецитації Їдама
рецитуй цю мантру разом з голосними та приголосними
та мантрою взаємозалежного виникнення.
Також рецитуй це час від часу над їжею.
\normalsize
