\newpage
\section{Чотири думки, що відволікають \\ від самсари}
\\
\subsection{Свободи та переваги людського народження}
\\
\subsection*{Вісім свобод}
\\
\ti
ད་རེས་དམྱལ་བ་ཡི་དྭགས་དུད་འགྲོ་དང༌།\\
ཚེ་རིང་ལྷ་དང་ཀླ་ཀློ་ལོག་ལྟ་ཅན།\\
སངས་རྒྱས་མ་བྱོན་ཞིང་དང་ལྐུགས་པ་སྟེ།\\
མི་ཁོམས་བརྒྱད་ལས་ཐར་བའི་དལ་བ་ཐོབ།\\
\\
\ru
Родитися в аду, голодним духом чи твариною,\\
Серед богів вічноживих чи дикою людиною,\\
У світі, куди Будда не прийшов, родитись слабоумним -\\
Я вільних від цих станів восьми неблагоумних,\\
Немає там нагоди вірить в карму\\
Й практикувати Дхарму.\\

\newpage
\subsection*{П'ять власних дарів і п'ять дарів оточення}
\\
\ti
མིར་གྱུར་དབང་པོ་ཚང་དང་ཡུལ་དབུས་སྐྱེས།\\
ལས་མཐའ་མ་ལོག་བསྟན་ལ་དད་པ་སྟེ།\\
རང་ཉིད་འབྱོར་པ་ལྔ་ཚང་སངས་རྒྱས་བྱོན། \\
ཆོས་གསུངས་བསྟན་པ་གནས་དང་དེ་ལ་ཞུགས།\\
བཤེས་གཉེན་དམ་པས་ཟིན་དང་གཞན་འབྱོར་ལྔ།\\
ཐམས་ཅད་རང་ལ་ཚང་བའི་གནས་ཐོབ་ཀྱང༌།\\
རྐྱེན་མང་ངེས་པ་མེད་པས་ཚེ་སྤངས་ནས།\\
འཇིག་རྟེན་ཕ་རོལ་ཉིད་དུ་སོན་པར་འགྱུར།\\
བློ་སྣ་ཆོས་ལ་སྒྱུར་ཅིག་གུ་རུ་མཁྱེན།\\
ལམ་གོལ་དམན་པར་མ་གཏོང་ཀུན་མཁྱེན་རྗེ།\\
གཉིས་སུ་མེད་དོ་དྲིན་ཅན་བླ་མ་མཁྱེན།\\
\\
\ru
Народження людиною, повнота власних органів чуття,\\
В країні серединній, достойний спосіб власного життя,\\
У Вчення тверда віра --- це п'ять твоїх дарів.\\
Поява Будди, Дхарми проповідь та існування Вчення,\\
Духовного Святого Друга підтримка і наслідування Вчення ---\\
Це п'ять оточення дарів.\\
\\
Дарами володію, та, загубив їх з власної вини\\
Через обставини неясні, відправився у інші народження й світи.\\
\\
Всезнаючий Гуру, зверни мої думки до Дхарми.\\
Всезнаючий Владико, на низький путь не дай піти у найми.\\
Ти Вчитель нероздільний, пам`ятай про мене!\\

\newpage
\subsection*{Трудність здобуття}
\ti ད་རེས་དལ་རྟེན་དོན་ཡོད་མ་བྱས་ན།\\
ཕྱི་ནས་ཐར་པ་བསྒྲུབ་པའི་རྟེན་མི་རྙེད།\\
བདེ་འགྲོའི་རྟེན་ལ་བསོད་ནམས་ཟད་གྱུར་ནས།\\
ཤི་བའི་འོག་ཏུ་ངན་སོང་ངན་འགྲོར་འཁྱམས།\\
དགེ་སྡིག་མི་ཤེས་ཆོས་ཀྱི་སྒྲ་མི་ཐོས།\\
དགེ་བའི་བཤེས་དང་མི་མཇལ་མཚང་རེ་ཆེ།\\
སེམས་ཅན་ཙམ་གྱི་གྲངས་དང་རིམ་པ་ལ།\\
བསམས་ན་མི་ལུས་ཐོབ་པ་སྲིད་མཐའ་ཙམ།\\
མི་ཡང་ཆོས་མེད་སྡིག་ལ་སྤྱོད་མཐོང་ན།\\
ཆོས་བཞིན་སྤྱོད་པ་ཉིན་མོའི་སྐར་མ་ཙམ།\\
བློ་སྣ་ཆོས་ལ་སྒྱུར་ཅིག་གུ་རུ་མཁྱེན།\\
ལམ་གོལ་དམན་པར་མ་གཏོང་ཀུན་མཁྱེན་རྗེ།\\
གཉིས་སུ་མེད་དོ་དྲིན་ཅན་བླ་མ་མཁྱེན།\\
\\
\ru
Якщо не скористаюся своїми перевагами й здобутками ---
Не зможу я знайти опору для досягнення свободи у майбутньому.
Розтративши заслуги всі щасливого народження і благих днів,
По смерті я блукатиму в дурних народженнях приземлених світів. \\
\\
Не розрізню чесноти й нечесноти, звучання чисте Дхарми не почую,
Духовного я Друга не зустріну, о, як жахливо я усе зруйную!
Народження людське --- це винятковість. Та, живучи серед людей пітьми,
Багато їх погрозою в нечеснотах, і рідко хто за Дхармою живе, як вдень зірки.\\
\\
Всезнаючий Гуру, зверни думки мої до Дхарми, Всезнаючий Владико,
Не дай зійти на путь низький, мій Нероздільний Милосердний Вчитель!\\

\newpage
\subsection*{Вісім непередбачуваних обставин}
\\
\ti
གལ་ཏེ་མི་ལུས་རིན་ཆེན་གླིང་ཕྱིན་ཡང༌།\\
ལུས་རྟེན་བཟང་ལ་བྱུར་པོ་ཆེ་ཡི་སེམས། \\
ཐར་པ་བསྒྲུབ་པའི་རྟེན་དུ་མི་རུང་ཞིང༌། \\
ཁྱད་པར་བདུད་ཀྱིས་ཟིན་དང་དུག་ལྔ་འཁྲུགས། \\
ལས་ངན་ཐོག་ཏུ་བབས་དང་ལེ་ལོས་གཡེངས། \\
གཞན་ཁོལ་བྲན་གཡོག་འཇིགས་སྐྱོབ་ཆོས་ལྟར་བཅོས། \\
རྨོངས་སོགས་འཕྲལ་བྱུང་རྐྱེན་གྱི་མི་ཁོམ་བརྒྱད། \\
བདག་ལ་ཆོས་ཀྱི་འགལ་ཟླར་ལྷགས་པའི་ཚེ། \\
བློ་སྣ་ཆོས་ལ་བསྒྱུར་ཅིག་གུ་རུ་མཁྱེན། \\
ལམ་གོལ་དམན་པར་མ་གཏོང་ཀུན་མཁྱེན་རྗེ། \\
གཉིས་སུ་མེད་དོ་དྲིན་ཅན་བླ་མ་མཁྱེན། \\
\\
\ru
\noindent
Хоч я прибув на найцінніший острів, народження людини,
В досягненні свободи хиткий розум не дійде й середини.\\
\\
Під владою умов і п'яти ядів, що тиснуть перешкодами,
Відволікають лінню й неблагою кармою, як раб під кодами,
Зі страху Дхарму вчу, вдаю, що практикую, невігластво ---
Обставин вісім роблять Дхарми практику без сенсу, некрасивою. \\
\\
Всезнаючий Гуру, зверни думки мої до Дхарми, Всезнаючий Владико,
Не дай зійти на путь низький, мій Нероздільний Милосердний Вчитель!\\

\newpage
\subsection*{Вісім несумісних з Дхармою обставин}
\\
\ti
སྐྱོ་ཤས་ཆུང་ཞིང་དད་པའི་ནོར་དང་བྲལ། \\
འདོད་སྲེད་ཞགས་པས་བཅིངས་དང་ཀུན་སྤྱོད་རྩུབ། \\
མི་དགེ་སྡིག་ལ་མི་འཛེམ་ལས་མཐའ་ལོག \\
རིས་ཆད་བློ་ཡི་མི་ཁོམ་རྣམ་པ་བརྒྱད། \\
བདག་ལ་ཆོས་ཀྱི་འགལ་ཟླར་ལྷགས་པའི་ཚེ། \\
བློ་སྣ་ཆོས་ལ་བསྒྱུར་ཅིག་གུ་རུ་མཁྱེན། \\
ལམ་གོལ་དམན་པར་མ་གཏོང་ཀུན་མཁྱེན་རྗེ། \\
གཉིས་སུ་མེད་དོ་དྲིན་ཅན་བླ་མ་མཁྱེན། \\
\\
\ru
Слабке відречення, неусвідомлення твердої віри цінності,\\
Прив'язаність до світу марна, схильність до розпусти і невірності,
Неблагочесна насолода та шкідливі дії через край,
Відсутність інтересу та порушення обітів і самай, ---
Це вісім станів розуму, що несумісні з Дхармою, як лід у спеці.\\
\\
Коли вони приходять --- Дхарми практика моя у небезпеці!\\
\\
Всезнаючий Гуру, зверни думки мої до Дхарми, Всезнаючий Владико,
Не дай зійти на путь низький, мій Нероздільний Милосердний Вчитель!\\

\newpage
\subsection{Непостійність життя}
\ti
ད་ལྟ་ནད་དང་སྡུག་བསྔལ་གྱིས་མ་གཟིར།\\
བྲན་ཁོལ་ལ་སོགས་གཞན་དབང་མ་གྱུར་པས།\\
རང་དབང་ཐོབ་པའི་རྟེན་འབྲེལ་འགྲིག་དུས་འདིར།\\
སྙོམས་ལས་ངང་དུ་དལ་འབྱོར་ཆུད་གསོན་ན།\\
འཁོར་དང་ལངོ ས་སྤྱོད་ཉེ་དུ་འབྲེལ་བ་ལྟ།\\
ལྟ་ཅི་གཅེས་པར་བཟུང་བའི་ལུས་འདི་ཡང་།\\
མལ་གྱི་ནང་ནས་ས་ཕྱོགས་སྟོང་པར་བསྐྱལ།\\
ཝ་དང་བྱ་རྒོད་ཁྱི་ཡིས་འདྲད་པའི་དུས།\\
བར་དོའི་ཡུལ་ན་འཇིགས་པ་ཤིན་ཏུ་ཆེ།\\
བློ་སྣ་ཆོས་ལ་བསྒྱུར་ཅིག་གུ་རུ་མཁྱནེ།\\
ལམ་གོལ་དམན་པར་མ་གཏོང་ཀུན་མཁྱནེ་རྗེ།\\
གཉིས་སུ་མེད་དོ་དྲིན་ཅན་བླ་མ་མཁྱེན།\\
\\
\ru
Тепер я сповнений стражданнями й хворобами сповна,
Але під владу інших не потрапив я, подібно до раба.\\
\\
Тому коли я, маючи предосконалий успіх незалежності,
Людське життя розтрачу через лінь без цілі із необережності,
Коли скарби прийдеться залишити і близьких, рідню, коханих,
Коли це випещене тіло переселиться з землі жаданих
В пустинне місце для розшматування лисам, стерв'ятникам, собакам,
Тоді в бардо вже не залишиться нічого, окрім страху й жаху.\\
\\
Всезнаючий Гуру, зверни думки мої до Дхарми, Всезнаючий Владико,
Не дай зійти на путь низький, мій Нероздільний Милосердний Вчитель!\\

\newpage
\subsection{Закон причини та наслідку безпомилковий}
\\
\ti
དགེ་སྡིག་ལས་ཀྱི་རྣམ་སྨིན་ཕྱི་བཞིན་འབྲང་།\\
\\
\ru
Наслідки благочесних і неблагочесних дій\\
Завжди ідуть за мною, як бджолиний рій.\\

\subsection{Недоліки самсари}

\subsection*{Страждання у гарячих адах}
\\
\ti
ཁྱད་པར་དམྱལ་བའི་འཇིག་རྟེན་ཉིད་སོན་ན།\\
ལྕགས་བསྲེགས་ས་གཞིར་མཚོན་གྱིས་མགོ་ལུས་འདྲལ།\\
སོག་ལེས་གཤོགས་དང་ཐོ་ལུམ་འབར་བས་འཚིར།\\
སྒོ་མེད་ལྕགས་ཁྱམི་འཐུམས་པར་འོ་དོད་འབོད།\\
འབར་བའི་གསལ་ཤིང་གིས་འབུགས་ཁྲོ་ཆུར་འཚོད།\\
ཀུན་ནས་ཚ་བའི་མེས་བསྲེགས་བརྒྱད་ཚན་གཅིག\\
\\
\ru
Якщо вже народжуся я лихим створінням адів,\\
Обезголовлений і зранений я буду на поверхні лави,\\
Розсічений пилою і роздроблений гарячим молотом,\\
Зі страху буду вити, оточений залізним мороком,\\
Пронизаний гарячим списом, зварений у лаві,\\
Спалений полум'ям, пізнаю я весь жах восьми гарячих адів.\\
\\

\newpage
\subsection*{Cтраждання в холодних адах}
\\
\ti
གངས་རི་སྟུག་པོའི་འདབས་དང་ཆུ་འཁྱགས་ཀྱི། \\
གཅོང་རོང་ཡ་ངའི་གནས་སུ་བུ་ཡུག་སྦྲེབས། \\
གྲང་རེག་རླུང་གིས་བཏབ་པའི་ལང་ཚོ་ནི། \\
ཆུ་བུར་ཅན་དང་ལྷག་པར་བརྡོལ་བ་ཅན། \\
སྨྲེ་སྔགས་རྒྱུན་མི་ཆད་པར་འདོན་པ་ཡང་། \\
ཚོར་བའི་སྡུག་བསྔལ་བརྣག་པར་དཀའ་བ་ཡིས།\\
ཟུངས་ཀྱིས་རབ་བཏང་འཆི་ཁའི་ནད་པ་བཞིན།\\
ཤུགས་རིང་འདོན་ཅིང་སོ་ཐམ་པགས་པ་འགས། \\
ཤའུ་ཐོན་ནས་ལྷག་པར་འགས་ཏེ་བརྒྱད།\\
\\
\ru
Там, на гірських вершинах, біля обривів льодяних,\\
Посеред місць страшних, де заметіль та білий сніг,\\
Моє преніжне тіло стрічає вітер, що не стих,\\
Його у гнійні язви й рани трансформує гріх.\\
То безперервний плач, стражденний крик людини,\\
Котру в немислимих стражданнях покидають сили.\\
Глибоко дихаю, сціпивши зуби до кісток.\\
Плоть тлі заживо, не даючи піти, хоча до смерті крок.\\
Пізнати відчай весь восьми холодних вдів то урок.\\

\newpage
\subsection*{Cтраждання в проміжних адах}
\\
\ti
དེ་བཞིན་སྤུ་གྲིའི་ཐང་ལ་རྐང་པ་གཤོགས།\\
རལ་གྲིའི་ཚལ་དུ་ལུས་ལ་བཅད་གཏུབས་བྱེད།\\
རོ་མྱགས་འདམ་ཚུད་ཐལ་ཚན་རབ་མེད་ཀློང་། \\
མནར་བའི་ཉེ་འཁོར་བ་དང་འགྱུར་བ་ཅན།\\
\\
\ru
Стрічками розрізано ступні в Рівнині із Лез,\\
У Лісі Мечів наше тіло шматують ножами,\\
Затягує Яма Вугілля, Болото Трупарні без меж, -\\
Ось Адів Проміжних страждання трапляються з нами.\\

\subsection*{Cтраждання в невизначених адах}
\\
\ti
སྒོ་དང་ཀ་བ་ཐབ་དང་ཐག་པ་སོགས།\\
རྟག་ཏུ་བཀོལ་ཞིང་སྤྱོད་པའི་ཉི་ཚེ་བ། \\\
རྣམ་གྲངས་བཅོ་བརྒྱད་གང་ལས་འབྱུང་བའི་རྒྱུ། \\
ཞེ་སྡང་དྲག་པོའི་ཀུན་སློང་སྐྱེས་པའི་ཚེ། \\
བློ་སྣ་ཆོས་ལ་བསྒྱུར་ཅིག་གུ་རུ་མཁྱནེ། \\
ལམ་གོལ་དམན་པར་མ་གཏོང་ཀུན་མཁྱནེ་རྗེ། \\
གཉིས་སུ་མེད་དོ་དྲིན་ཅན་བླ་མ་མཁྱེན།\\
\\
\ru
Дверми, колонами, мотузками та піччю --- чим завгодно я стаю,\\
Й мінливого я аду рабство пізнаю.\\
Причиною народження у вісімнадцяти адах\\
Є сильний гнів і ненависть моя, це жах.\\
\\
Всезнаючий Гуру, зверни думки мої до Дхарми, Всезнаючий Владико,
Не дай зійти на путь низький, мій Нероздільний Милосердний Вчитель!\\

\newpage
\subsection*{Cтраждання у світі голодних духів}
\ti
དེ་བཞིན་ཕོངས་ལ་ཉམས་མི་དགའ་བའི་ཡུལ།\\
བཟའ་བཏུང་ལངོ ས་སྤྱོད་མིང་ཡང་མི་གྲགས་པར། \\
ཟས་སྐོམ་ལོ་ཟླར་མི་རྙེད་ཡི་དྭགས་ལུས། \\
རིད་ཅིང་ལྡང་བའི་སྟོབས་ཉམས་རྣམ་པ་གསུམ། \\
གང་ལས་འབྱུང་བའི་རྒྱུ་ནི་སེར་སྣ་ཡིན།\\
\\
\ru
У темряві й безодні, де про комфорт і воду не чули й поготів,
Там прети не знаходять питво і їжу впродовж довгих місяців й років.
Тіла настільки виснажені в них, що навіть не стоять.
Вони страждають від трьох видів забруднень і заклять.
Подібне переродження --- це жадібності результат.

\subsection*{Cтраждання у світі тварин}
\ti
གཅིག་ལ་གཅིག་བཟའ་གསོད་པའི་འཇིགས་པ་ཆེ།\\
བཀོལ་ཞིང་སྤྱོད་པས་ཉམ་ཐག་བླང་དོར་རྨོངས། \\
ཕ་མཐའ་མེད་པའི་སྡུག་བསྔལ་གྱིས་གཟིར་བའི། \\
ས་བོན་གཏི་མུག་མུན་པར་འཁྱམས་པ་བདག \\
བློ་སྣ་ཆོས་ལ་བསྒྱུར་ཅིག་གུ་རུ་མཁྱནེ། \\
ལམ་གོལ་དམན་པར་མ་གཏོང་ཀུན་མཁྱནེ་རྗེ། \\
གཉིས་སུ་མེད་དོ་དྲིན་ཅན་བླ་མ་མཁྱེན།\\
\\
\ru
В постійнім страсі, що тебе хтось з'їсть, утомлені від рабства,
Не знаючи, що прийняти, а що відкинути, де знак шаманства,
Тварини, зв'язані стражданням нескінченним, суть якого --- вперта тупість.
Коли я потрапляють в темряву невідання --- безвихідь бачу, глухість.\\
\\
Всезнаючий Гуру, зверни думки мої до Дхарми, Всезнаючий Владико,
Не дай зійти на путь низький, мій Нероздільний Милосердний Вчитель!

\section*{Осмислення власних помилок}
\subsection*{Три колісниці}
\\
\ti
ཆོས་ལམ་ཞུགས་ཀྱང་ཉེས་སྤྱོད་མི་སྡོམ་ཞིང་།\\
ཐེག་ཆེན་སྒོར་ཞུགས་གཞན་ཕན་སེམས་དང་བྲལ།\\
དབང་བཞི་ཐོབ་ཀྱང་བསྐྱེད་རྫོགས་མི་སྒོམ་པའི།\\
ལམ་གོལ་འདི་ལས་བླ་མས་བསྒྲལ་དུ་གསོལ།\\
\\
\ru
Хоч я ступив на Дхарми путь, та помилок не зрікся я, між іншим.
Хоч увійшов у браму Махаяни, не маю наміру творити благо іншим.
Посвячення чотири я отримав, але не практикую всі стадії зародження й завершення.
О Вчителю, звільни мене від цього помилкового шляху збезчещення!\\

\subsection*{Світогляд, медитація та поведінка}
\\
\ti
ལྟ་བ་མ་རྟོགས་ཐོ་ཅོའི་སྤྱོད་པ་ཅན།\\
སྒོམ་པ་ཡངེ ས་ཀྱང་གོ་ཡུལ་འུད་གོག་\\
འཐག སྤྱོད་པ་ནོར་ཀྱང་རང་སྐྱོན་མི་སེམས་པའི། \\
ཆོས་དྲེད་འདི་ལས་བླ་མས་བསྒྲལ་དུ་གསོལ།\\
\\
\ru
Світогляд я не втілив у життя, веду себе як Йогин Мудрості Безумної.
Без стійкості та медитації собі я дозволя застигнути в концепціях та грі ума.
Хоч поведінка моя помилкова, та про себе не маю думки я осудної.
О Вчителю, звільни мене від впертості й високомірності, байдужості і зла!\\

\newpage
\subsection*{Відволікання від справжнього життя}
\\
\ti
ནང་པར་འཆི་ཡང་གནས་གོས་ནོར་ལ་སྲེད།\\
ན་ཚོད་ཡོལ་ཡང་ངེས་འབྱུང་སྐྱོ་ཤས་བྲལ། \\
ཐོས་པ་ཆུང་ཡང་ཡོན་ཏན་ཅན་དུ་རློམ། \\
མ་རིག་འདི་ལས་བླ་མས་བསྒྲལ་དུ་གསོལ།\\
\\
\ru
Померти можу завтра, але чіпляюсь за житла й багатства чари.
Хоч я немолодий, але позбавлений розчарування й зречення Самсари.
Пізнавши краплю Дхарми, я хвастаюсь усім в своїй ученості.
О Вчителю, звільни мене же від подібної обмеженості!\\

\subsection*{Вісім світських клопотів}
\\
\ti
རྐྱེན་ཁར་འཆོར་ཡང་འདུ་འཛི་གནས་སྐོར་སེམས།\\
དབེན་པ་བརྟེན་ཀྱང་རང་རྒྱུད་ཤིང་ལྟར་རེངས། \\
དུལ་བར་སྨྲ་ཡང་ཆགས་སྡང་མ་ཞིག་པའི། \\
ཆོས་བརྒྱད་འདི་ལས་བླ་མས་བསྒྲལ་དུ་གསོལ། \\
གཉིད་འཐུག་འདི་ལས་མྱུར་དུ་སད་དུ་གསོལ། \\
ཁྲི་མུན་འདི་ལས་མྱུར་དུ་དབྱུང་དུ་གསོལ། \\
ཞེས་འབོད་པ་དྲག་པོས་ཐུགས་རྗེ་བསླང་བར་བྱའོ།\\
\\
\ru
Хоч небезпекам піддаюсь, та Дхарму я шукаю,\\
Місця для спілкування із людьми паломництвом вважаю.\\
В ретриті я, свідомість --- посох, не розмина півкулю ліву.\\
Хвалюся спокоєм, але не вільний я від жадібності й гніву.\\
О Вчителю, від восьми цілей світських мене ти захисти!\\
Зі сну глибокого невігластва мене ти пробуди!\\
Із цього темного самозаточення мене звільни!\\
