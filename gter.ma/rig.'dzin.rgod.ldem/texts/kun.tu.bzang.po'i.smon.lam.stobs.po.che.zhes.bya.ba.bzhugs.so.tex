% HEVEA % Copyright (c) 2015 Synrc Research Center

%\usepackage{afterpage}
%\usepackage{graphicx}
%\usepackage{tocloft}
\usepackage[english,russian]{babel}
\usepackage{fontspec}
%\usepackage{polyglossia}
\usepackage{hyphenat}
%\usepackage{caption}
%\usepackage[usenames,dvipsnames]{color}
\usepackage[top=18mm, bottom=25.4mm,
            inner=15mm,outer=18mm,
            paperwidth=142mm, paperheight=200mm]{geometry}

\hyphenation{бес-чувстве-нен cуще-ствования
 буду-щем разно-образии Дхар-му наско-лько счастли-вого ниж-них
 сохрани-лось всеведую-щие свер-нуть неза-висимости}

\fontencoding{T1}

%\setlength{\cftsubsecnumwidth}{2.5em}
%\defaultfontfeatures{Ligatures=TeX}

% include image for HeVeA and LaTeX

%\makeatletter
%\def\@seccntformat#1{\llap{\csname the#1\endcsname\hskip0.7em\relax}}
%\makeatother

\addto\captionsrussian{\renewcommand{\contentsname}{Зміст}}

\newcommand{\includeimage}[2]
{\begin{figure}[h!]
\centering
\includegraphics[width=\textwidth]{#1}
\caption{#2}
\end{figure}
}

\newcommand*{\titleNENDRO}
{
\newfontfamily{\cyrillicfont}{Geometria}
%\setdefaultlanguage{tibetan}
\setmainfont{Geometria}
    \begingroup
        \thispagestyle{empty}
        \hspace*{0.15\textwidth}
        \rule{1pt}{\textheight}
        \hspace*{0.05\textwidth}
        {
        \parbox[c][][s]{0.75\textwidth}
        {
             \vspace{-15cm}
%            \setdefaultlanguage{tibetan}
            \setmainfont{DDC Uchen}
             \textsc{\Large  ༄༅༎ །རྫོགས་པ་ཆེན་པོ་ཀློང་ཆེན་སྙིང་ཏིག་\\[0.05\baselineskip]
                                  གི་སྔོན་འགྲོའི་ངག་འདོན་ཁྲིགས་\\[0.05\baselineskip]
                                  སུ་བསྡེབས་པ་རྣམ་མཁྱེན་\\[0.05\baselineskip]
                                  ལམ་བཟང་བྱ་བ་བཞུགས་སོ། \\}
            \\
             \textsc{\noindent
%             \setdefaultlanguage{russian}
             \setmainfont{Geometria}
             Додрубчен Джигме Трінлей Озер \\ [0.3\baselineskip]
             \\
             \Large
             Неперевершений шлях \\[0.3\baselineskip]
             для пізнання зразу всього: \\[0.3\baselineskip]
             попередні практики \\[0.3\baselineskip]
             сердечної cутності \\[0.3\baselineskip]
             обширного простору \\[0.3\baselineskip]
             остаточного довершення.} \\[0.3\baselineskip]
        }}
    \endgroup
%    \setdefaultlanguage{russian}
    \setmainfont{Geometria}
}

% define images store

% start each section from new page

%\let\stdsection\section
%\renewcommand\section{\newpage\stdsection}

% define style for code listings

\lefthyphenmin=1
\hyphenpenalty=100
\tolerance=3000

%\newcommand\blankpage{
%    \null
%    \thispagestyle{empty}
%    \newpage}

\newcommand{\ru}{
%    \setdefaultlanguage{russian}
    \setmainfont{Geometria}
}

\newcommand{\ti}{
 %   \setdefaultlanguage{tibetan}
    \setmainfont{DDC Uchen}
}

%\newtoggle{russian@scriptlangequal}
%\newtoggle{tibetan@scriptlangequal}

\newlength\tindent
\setlength{\tindent}{\parindent}
\setlength{\parindent}{0pt}
\renewcommand{\indent}{\hspace*{\tindent}}

% HEVEA \begin{document}
% nyingma_author =Намкха Кхандро=
% HEVEA \title{Кунзанг Монлам}

\Section{Рігдзін Годем: Кунзанг Монлам}
\SectionNo{{\ti ༄༅། །ཀུན་ཏུ་བཟང་པོའི་སྨོན་ལམ་སྟོབས་པོ་ཆེ་ཞེས་བྱ་བ་བཞུགས་སོ།}}\\
\\
{\ti
ཙིཏྟ་ཨ༔ དེ་ནས་ཐོག་མའི་སངས་རྒྱས་ཀུན་ཏུ་བཟང་པོས༔\\
འཁོར་བའི་སེམས་ཅན་སངས་མི་རྒྱ་བའི་དབང་མེད་པའི་སྨོན་ལམ་ཁྱད་པར་ཅན་འདི་གསུང་ངོ༔}\\
\\
ЦІТТА А:

Після цього первісний Будда Самантабхадра вимовив 
це особливе устремління, через яке істоти в самсарі
не зможуть не просвітлитися.\\
\\
{\ti
ཧོ༔ སྣང་སྲིད་འཁོར་འདས་ཐམས་ཅད་ཀུན༔\\
གཞི་གཅིག་ལམ་གཉིས་འབྲས་བུ་གཉིས༔ \\
རིག་དང་མ་རིག་ཆོ་འཕྲུལ་ཏེ༔ \\
ཀུན་ཏུ་བཟང་པོའི་སྨོན་ལམ་གྱིས༔ \\
ཐམས་ཅད་ཆོས་དབྱིངས་ཕོ་བྲང་དུ༔ \\
མངོན་པར་རྫོགས་ཏེ་སངས་རྒྱས་ཤོག༔ }\\
\\
Хо! У всього явленого і сущого, у сансари і нірвани, \\
Є одна основа, два шляхи та два плоди, \\
Вони маніфестуються чарівним чином через усвідомлення та ігнорування реальності. \\
Тож нехай за допомогою цієї молитви Самантабхадри \\
Усі досягнуть явленого та досконалого просвітлення \\
У палаці головного простору феноменів. \\
\newpage
\\
{\ti
ཀུན་གྱི་གཞི་ནི་འདུས་མ་བྱས༔\\
རང་བྱུང་ཀློང་ཡངས་བརྗོད་དུ་མེད༔\\
འཁོར་འདས་གཉིས་ཀའི་མིང་མེད་དོ༔\\
དེ་ཉིད་རིག་ན་སངས་རྒྱས་ཏེ༔\\
མ་རིག་སེམས་ཅན་འཁོར་བར་འཁྱམས༔\\
ཁམས་གསུམ་སེམས་ཅན་ཐམས་ཅད་ཀྱིས༔\\
བརྗོད་མེད་གཞི་དོན་རིག་པར་ཤོག༔}\\
\\
Основа всього буття не створена; \\
Цьому самосущому, безмежному та невимовному простору \\
Невідомі назви «сансара» та «нірвана». \\
Ті, хто усвідомлюють це, є буддами, \\
А ті, хто не знають цього --- блукають сансарою як наділені помилковим розумом істоти. \\
То нехай усі істоти трьох світів \\
Усвідомлять невимовну суть основи буття. \\
\\
{\ti
ཀུན་ཏུ་བཟང་པོ་ང་ཡིས་ཀྱང༔ \\
རྒྱུ་རྐྱེན་མེད་པ་གཞི་ཡི་དོན༔ \\
དེ་ཉིད་གཞི་ལས་རང་བྱུང་རིག༔ \\
ཕྱི་ནང་སྒྲོ་སྐུར་སྐྱོན་མ་བཏགས༔ \\
དྲན་མེད་མུན་པའི་དྲི་མ་བྲལ༔ \\
དེ་ཕྱིར་རང་སྣང་སྐྱོན་མ་གོས༔ \\
རང་རིག་སོ་ལ་གནས་པ་ལ༔}\\
\\
Я, Самантабхадра, збагнув суть безпричинної та необумовленої \\
Основи, яка є самовиникаючою присутністю, \\
І перебуваю в самоусвідомленні, чистому від: \\
Недоліків самопрояву, неповноцінності зовнішніх і внутрішніх ментальних крайнощів \\
І мороку несвідомості; \\
\\
\newpage
{\ti
སྲིད་གསུམ་འཇིག་ཀྱང་སྔངས་སྐྲག་མེད༔\\
སྣང་སེམས་གཉིས་སུ་མེད་པ་ལ༔\\
འདོད་ཡོན་ལྔ་ལ་ཆགས་པ་མེད༔\\
རྟོག་མེད་ཤེས་པ་རང་བྱུང་ལ༔\\
གདོས་པའི་གཟུགས་དང་ཁ་དོག་མེད༔}\\
\\
Навіть якщо руйнуватимуться три світи, мені нічого боятися \\
У недвоїстості проявів та розуму \\
Немає прихильності до п'яти чуттєвих задоволень. \\
У неконцептуальному, самосущому знанні \\
Немає матеріальних форм та п'яти емоційних отрут. \\
\\
{\ti
རིག་པའི་གསལ་ཆ་མ་འགགས་པས༔ \\
ངོ་བོ་གཅིག་ལ་ཡེ་ཤེས་ལྔ༔ \\
ཡེ་ཤེས་ལྔ་པོ་སྨིན་པ་ལས༔ \\
ཐོག་མའི་སངས་རྒྱས་རིགས་ལྔ་བྱུང༔}\\
\\
З безперервного аспекту ясності усвідомлення \\
З'являються п'ять видів мудрості, які мають єдину сутність. \\
Дозріваючи, ці п'ять видів мудрості \\
Стають п'ятьма первісними сімействами будд. \\
\\
{\ti
དེ་ལས་ཡེ་ཤེས་མཐའ་རྒྱས་པས༔ \\
སངས་རྒྱས་བཞི་བཅུ་རྩ་གཉིས་བྱུང༔ \\
ཡེ་ཤེས་ལྔ་ཡི་རྩལ་ཤར་བས༔ \\
ཁྲག་འཐུང་དྲུག་ཅུ་ཐམ་པ་བྱུང༔}\\
\\
Завдяки поширенню цієї первісної мудрості \\
З'являються 42 мирні будди, \\
А внаслідок сходження енергії п'яти видів мудрості \\
Виникають 60 гнівних будд, які п'ють кров емоцій. \\
\\
\newpage
\\
{\ti
དེ་ཕྱིར་གཞི་རིག་འཁྲུལ་མ་མྱོང༔ \\
ཐོག་མའི་སངས་རྒྱས་ང་ཡིན་པས༔ \\
ང་ཡི་སྨོན་ལམ་བཏབ་པ་ཡིས༔ \\
ཁམས་གསུམ་འཁོར་བའི་སེམས་ཅན་གྱིས༔ \\
རང་བྱུང་རིག་པ་ངོ་ཤེས་ནས༔ \\
ཡེ་ཤེས་ཆེན་པོ་མཐའ་རྒྱས་ཤོག༔}\\
\\
Тому основа, як усвідомлення, ніколи не знала помилки. \\
Я, Самантабхадра, первісний будда \\
Підношу молитву, щоб \\
Усі істоти, що блукають по трьох світах сансари, \\
Збагнули самосвідоме усвідомлення \\
І щоб велика мудрість дійшла до країв світу. \\
\\
{\ti
ང་ཡི་སྤྲུལ་པ་རྒྱུན་མི་ཆད༔ \\
བྱེ་བ་ཕྲག་བརྒྱ་བསམ་ཡས་འགྱེད༔ \\
གང་ལ་གང་འདུལ་སྣ་ཚོགས་སྟོན༔ \\
ང་ཡི་ཐུགས་རྗེ་སྨོན་ལམ་གྱིས༔ \\
ཁམས་གསུམ་འཁོར་བའི་སེམས་ཅན་ཀུན༔ \\
རིགས་དྲུག་གནས་ནས་འཐོན་པར་ཤོག༔}\\
\\
Безперервним потоком мої еманації \\
Поширюватимуться у міріадах незбагненних образів, \\
Представляючись у тих чи інших формах \\
Перед тими, хто потребує скорення. \\
Нехай завдяки моїй молитві, сповненій співчуття, \\
Усі істоти трьох світів самсари \\
Зникнуть із обителів шести світів. \\
\\
\newpage
{\ti
དང་པོ་སེམས་ཅན་འཁྲུལ་པ་རྣམས༔ \\
གཞི་ལ་རིག་པ་མ་ཤར་བས༔ \\
ཅི་ཡང་དྲན་མེད་ཐོམ་མེ་བ༔ \\
དེ་ཀ་མ་རིག་འཁྲུལ་པའི་རྒྱུ༔ \\
དེ་ལ་ཧད་ཀྱིས་བརྒྱལ་བ་ལ༔ \\
དངངས་སྐྲག་ཤེས་པ་ཟ་ཟིར་འགྱུས༔ \\
དེ་ལ་བདག་གཞན་དགྲར་འཛིན་སྐྱེས༔}\\
\\
Істоти помиляються, тому що на самому початку \\
У них не виникло усвідомлення щодо основи; \\
Через це у них відсутня якась усвідомленість. \\
Це і є причина помилки та незнання. \\
Через такий несвідомий стан \\
З'являються побоювання, страх і нервозність, \\
Від чого виникає чіпляння за себе та сприйняття інших у вигляді ворогів. \\
\\
{\ti
བག་ཆགས་རིམ་གྱིས་བརྟས་པ་ལས༔ \\
འཁོར་བ་ལུགས་སུ་འཇུག་པ་བྱུང༔ \\
དེ་ལས་ཉོན་མོངས་དུག་ལྔ་རྒྱས༔ \\
དུག་ལྔའི་ལས་ལ་རྒྱུན་ཆད་མེད༔}\\
\\
Поступово така тенденція входить у звичку, \\
І з цього виникає ланцюжок самсари; \\
Розвиваються п'ять емоційних отрут, \\
Які спричиняють безперервний потік карми. \\
\\
\newpage
{\ti
དེ་ཕྱིར་སེམས་ཅན་འཁྲུལ་པའི་གཞི༔ \\
དྲན་མེད་མ་རིག་ཡིན་པའི་ཕྱིར༔ \\
སངས་རྒྱས་ང་ཡི་སྨོན་ལམ་གྱིས༔ \\
ཀུན་གྱིས་རིག་པ་རང་ཤེས་ཤོག༔}\\
\\
Тому незнання та відсутність усвідомленості \\
Є основою помилки істот, що відчувають; \\
Тож нехай завдяки моїй просвітленій молитві \\
Усі спіткають своє усвідомлення. \\
{\ti
ལྷན་ཅིག་སྐྱེས་པའི་མ་རིག་པ༔ \\
ཤེས་པ་དྲན་མེད་ཐོམ་མེ་བ༔ \\
ཀུན་ཏུ་བཏགས་པའི་མ་རིག་པ༔ \\
བདག་གཞན་གཉིས་སུ་འཛིན་པ་ཡིན༔ \\
ལྷན་སྐྱེས་ཀུན་བཏགས་མ་རིག་གཉིས༔ \\
སེམས་ཅན་ཀུན་གྱི་འཁྲུལ་གཞི་ཡིན༔}\\
\\
Супровідне невідання від ігнорування — \\
Це відволікання сприйняття та відсутність усвідомленості. \\
Концептуальне незнання — \\
Це двояке сприйняття — розподіл світу на «себе» та «інших». \\
Як супровідне, так і концептуальне незнання \\
Є основою оманливого мислення істот. \\
\newpage
{\ti
སངས་རྒྱས་ང་ཡི་སྨོན་ལམ་གྱིས༔ \\
འཁོར་བའི་སེམས་ཅན་ཐམས་ཅད་ཀྱི༔ \\
དྲན་མེད་འཐིབས་པའི་མུན་པ་སངས༔ \\
གཉིས་སུ་འཛིན་པའི་ཤེས་པ་དྭངས༔ \\
རིག་པ་རང་ངོ་ཤེས་པར་ཤོག༔}\\
\\
Тож нехай завдяки моїй просвітленій молитві \\
У всіх істот у сансарі \\
Розсіється похмура імла відсутності усвідомленості \\
Зникне подвійне сприйняття \\
І вони побачать природнє обличчя усвідомлення. \\
\\
{\ti
གཉིས་འཛིན་བློ་ནི་ཐེ་ཚོམ་སྟེ༔ \\
ཞེན་པ་ཕྲ་མོ་སྐྱེས་པ་ལས༔ \\
བག་ཆགས་མཐུག་པོ་རིམ་གྱིས་བརྟས༔ \\
ཟས་ནོར་གོས་དང་གནས་དང་གྲོགས༔ \\
འདོད་ཡོན་ལྔ་དང་བྱམས་པའི་གཉེན༔ \\
ཡིད་འོང་ཆགས་པའི་འདོད་པས་གདུང༔}\\
\\
Подвійна свідомість породжує сумніви; \\
Починаючи з появи тонких чіплянь, \\
Стереотипні тенденції поступово входять у звичку і набирають сили: \\
Їжа, багатство, одяг, дім та друзі, \\
П'ять чуттєвих задоволень, улюблені та близькі \\
Викликають болісну прихильність до приємних речей. \\
\\
\newpage
{\ti
དེ་དག་འཇིག་རྟེན་འཁྲུལ་པ་སྟེ༔ \\
གཟུང་འཛིན་ལས་ལ་ཟད་མཐའ་མེད༔ \\
ཞེན་པའི་འབྲས་བུ་སྨིན་པའི་ཚེ༔ \\
བརྐམ་ཆགས་གདུང་བའི་ཡི་དྭགས་སུ༔ \\
སྐྱེས་ནས་བཀྲེས་སྐོམ་ཡ་རེ་ང༔}\\
\\
Всі ці світські речі збивають зі шляху, \\
А карма, що виникає з двоїстого сприйняття, не має ні кінця, ні краю. \\
Коли визрівають плоди вподобань, \\
Істоти народжуються голодними духами, \\
Які страждають від бажань і потерпають від голоду та спраги. \\
\\
{\ti
སངས་རྒྱས་ང་ཡི་སྨོན་ལམ་གྱིས༔ \\
འདོད་ཆགས་ཞེན་པའི་སེམས་ཅན་རྣམས༔ \\
འདོད་པའི་གདུང་བ་ཕྱིར་མ་སྤངས༔ \\
འདོད་ཆགས་ཞེན་པ་ཚུར་མ་བླངས༔ \\
ཤེས་པ་རང་སོར་གློད་པ་ཡིས༔ \\
རིག་པ་རང་སོ་ཟིན་གྱུར་ནས༔ \\
ཀུན་རྟོག་ཡེ་ཤེས་ཐོབ་པར་ཤོག༔}\\
\\
Тож нехай за допомогою моєї просвітленої молитви \\
Усі мислячі істоти одержимі чіпляннями і вподобаннями \\
І ті, хто страждає від того, що не можуть розлучитися з бажаннями, \\
Перестануть здобувати і чіплятися. \\
Розкутнуть своє сприйняття в природному стані, \\
Нехай вони дозволять усвідомленню зайняти своє гідне становище \\
І спіткають мудрість, що все розрізняє. \\
\\
\newpage
{\ti
ཕྱི་རོལ་ཡུལ་གྱི་སྣང་བ་ལ༔ \\
འཇིགས་སྐྲག་ཤེས་པ་ཕྲ་མོ་འགྱུས༔ \\
སྡང་བའི་བག་ཆགས་བརྟས་པ་ལས༔ \\
དགྲར་འཛིན་བརྡེག་གསོད་རགས་པ་སྐྱེས༔ \\
ཞེ་སྡང་འབྲས་བུ་སྨིན་པའི་ཚེ༔ \\
དམྱལ་བའི་བཙོ་བསྲེག་སྡུག་རེ་བསྔལ༔}\\
\\
По відношенню до об'єктів, що сприймаються як зовнішні, \\
Виникає тонке відчуття побоювання. \\
Тенденція до їх відкидання стає все сильнішою, \\
І вороже сприйняття призводить до насильства та вбивств. \\
Коли ж визріває плід гніву та агресії, \\
Істоти народжуються в пеклі, страждаючи від того, що їх палять і варять. \\
\\
{\ti
སངས་རྒྱས་ང་ཡི་སྨོན་ལམ་གྱིས༔ \\
འགྲོ་དྲུག་སེམས་ཅན་ཐམས་ཅད་ཀྱི༔ \\
ཞེ་སྡང་དྲག་པོ་སྐྱེས་པའི་ཚེ༔ \\
སྤང་བླང་མི་བྱ་རང་སོར་གློད༔ \\
རིག་པ་རང་སོ་ཟིན་གྱུར་ནས༔ \\
གསལ་བའི་ཡེ་ཤེས་ཐོབ་པར་ཤོག༔}\\
\\
Тож нехай за допомогою моєї просвітленої молитви \\
Усі істоти шести світів, що зазнають сильного гніву та агресії, \\
У той момент, коли це з'являється в них, \\
зможуть розслабитися в природному стані, нічого не приймаючи і нічого не відкидаючи, \\
І спіткають світлоносну мудрість. \\
\\
\newpage
{\ti
རང་སེམས་ཁེངས་པར་གྱུར་པ་ལས༔ \\
གཞན་ལ་འགྲན་སེམས་སྨད་པའི་བློ༔ \\
ང་རྒྱལ་དྲག་པོའི་སེམས་སྐྱེས་པས༔ \\
བདག་གཞན་འཐབ་རྩོད་སྡུག་བསྔལ་སྤྱོད༔ \\
ལས་དེའི་འབྲས་བུ་སྨིན་པའི་ཚེ༔ \\
འཕོ་ལྟུང་མྱོང་བའི་ལྷ་རུ་སྐྱེས༔}\\
\\
Коли в умі з'являється зарозумілість, \\
Його супроводжує бажання змагатися та принижувати інших. \\
Така сильна гординя, захоплюючи розум, \\
Приводить до страждань від сварок та боротьби з іншими. \\
Коли визрівають плоди цієї карми, \\
Відбувається народження у світі богів, де страждають від змін, та падають (у нижчі світи). \\
{\ti
སངས་རྒྱས་ང་ཡི་སྨོན་ལམ་གྱིས༔ \\
ཁེངས་སེམས་སྐྱེས་པའི་སེམས་ཅན་རྣམས༔ \\
དེ་ཚེ་ཤེས་པ་རང་སོར་གློད༔ \\
རིག་པ་རང་སོ་ཟིན་གྱུར་ནས༔ \\
མཉམ་པ་ཉིད་ཀྱི་དོན་རྟོགས་ཤོག༔}\\
\\
Тож нехай за допомогою моєї просвітленої молитви \\
Всі істоти, що мислять, у яких з'являється зарозумілість, \\
У цей самий момент розкутають свою свідомість у природному стані. \\
І коли їх усвідомлення займе своє природнє становище, \\
Нехай вони реалізують суть рівносності. \\
\\
\newpage
{\ti
གཉིས་འཛིན་བརྟས་པའི་བག་ཆགས་ཀྱིས༔ \\
བདག་བསྟོད་གཞན་སྨོད་ཟུག་རྔུའི་ལས༔ \\
འཐབ་རྩོད་འགྲན་སེམས་བརྟས་པ་ལས༔ \\
གསོད་གཅོད་ལྷ་མིན་གནས་སུ་སྐྱེས༔ \\
འབྲས་བུ་དམྱལ་བའི་གནས་སུ་ལྟུང༔}\\
\\
Коли ми страждаємо від того, щоб підняти себе, принизивши при цьому інших, \\
У розумі, що суперничає, посилюється тенденція до сварок і боротьби; \\
З цієї причини відбувається народження у світі бійок, \\
А це, своєю чергою, веде до падіння в пекло. \\
\\
{\ti
སངས་རྒྱས་ང་ཡི་སྨོན་ལམ་གྱིས༔ \\
འགྲན་སེམས་འཐབ་རྩོད་སྐྱེས་པ་རྣམས༔ \\
དགྲར་འཛིན་མི་བྱ་རང་སོར་གློད༔ \\
ཤེས་པ་རང་སོ་ཟིན་གྱུར་ནས༔ \\
ཕྲིན་ལས་ཐོགས་མེད་ཡེ་ཤེས་ཤོག༔}\\
\\
Тож нехай завдяки моїй просвітленій молитві \\
Всі істоти, одержимі думками про суперництво, сварки та боротьбу, \\
Замість того, щоб сприймати інших вороже, \\
Розслабляться у природному стані; \\
Нехай їх сприйняття займе природне становище \\
І вони спіткають мудрість безперешкодної активності. \\
\\
\newpage
{\ti
དྲན་མེད་བཏང་སྙོམས་ཡེངས་པ་དང༔ \\
འཐིབས་དང་རྨུགས་དང་བརྗེད་པ་དང༔ \\
བརྒྱལ་དང་ལེ་ལོ་གཏི་མུག་པས༔ \\
འབྲས་བུ་སྐྱབས་མེད་བྱོལ་སོང་འཁྱམས༔}\\
\\
Внаслідок відсутності усвідомленості, апатії, відволікань, \\
Сонливості, притупленого сприйняття та забудькуватості, \\
Лінощі, тупості та несвідомого, відсутнього стану, \\
Стають безпорадними тваринами. \\
\\
{\ti
སངས་རྒྱས་ང་ཡི་སྨོན་ལམ་གྱིས༔ \\
གཏི་མུག་བྱིང་བའི་མུན་པ་ལ༔ \\
དྲན་པ་གསལ་བའི་མདངས་ཤར་ནས༔ \\
རྟོག་མེད་ཡེ་ཤེས་ཐོབ་པར་ཤོག༔}\\
\\
Тож нехай за допомогою моєї просвітленої молитви \\
У тих, хто перебуває у мороці безпроглядної тупості, \\
Зійде світлоносне сяйво свідомості \\
І вони спіткають неконцептуальну мудрість. \\
\\
\newpage
{\ti
ཁམས་གསུམ་སེམས་ཅན་ཐམས་ཅད་ཀྱང༔ \\
ཀུན་གཞི་སངས་རྒྱས་ང་དང་མཉམ༔ \\
དྲན་མེད་འཁྲུལ་པའི་གཞི་རུ་སོང༔ \\
ད་ལྟ་དོན་མེད་ལས་ལ་སྤྱོད༔\\
ལས་དྲུག་རྨི་ལམ་འཁྲུལ་པ་འདྲ༔}\\
\\
Усі істоти трьох світів \\
Рівнозначні мені, Будді вселенської основи буття, \\
Але через відсутність свідомості \\
вони опинилися в полоні помилки \\
І зараз займаються безглуздими справами. \\
\\
{\ti
ང་ནི་སངས་རྒྱས་ཐོག་མ་ཡིན༔ \\
འགྲོ་དྲུག་སྤྲུལ་པས་འདུལ་བའི་ཕྱིར༔ \\
ཀུན་ཏུ་བཟང་པོའི་སྨོན་ལམ་གྱིས༔ \\
སེམས་ཅན་ཐམས་ཅད་མ་ལུས་པ༔ \\
ཆོས་ཀྱི་དབྱིངས་སུ་སངས་རྒྱས་ཤོག༔}\\
\\
Шість видів карми – це помилка, яка відбувається як уві сні. \\
Я, первісний Будда, Присутній тут, щоб утихомирити шість видів істот через мої еманації. \\
Тож нехай завдяки молитві Самантабхадри \\
Усі істоти, що мислять, без винятку \\
Просвітляться в основному просторі явищ! \\
\\
\newpage
{\ti
ཨ་ཧོ༔ ཕྱིན་ཆད་རྣལ་འབྱོར་སྟོབས་ཅན་གྱིས༔ \\
འཁྲུལ་མེད་རིག་པ་རང་གསལ་ངང༔ \\
སྨོན་ལམ་སྟོབས་ཆེན་འདི་བཏབ་པས༔ \\
འདི་ཐོས་སེམས་ཅན་ཐམས་ཅད་ཀུན༔ \\
སྐྱེ་བ་གསུམ་ནས་མངོན་འཚང་རྒྱ༔}\\
\\
A ХО! \\
Якщо, починаючи з цього моменту, могутній йогін,
Перебуваючи у вільному від оман, самосяючому усвідомленні, \\
Вимовить це прагнення великої сили, \\
То всі істоти, які почують його, \\
Досягнуть явного просвітління протягом трьох життів. \\
\\
{\ti
ཉི་ཟླ་གཟའ་ཡིས་ཟིན་པ་འམ༔ \\
སྒྲ་དང་ས་གཡོ་འབྱུང་བ་འམ༔ \\
ཉི་མ་ལྡོག་འགྱུར་ལོ་འཕོའི་དུས༔ \\
རང་ཉིད་ཀུན་ཏུ་བཟང་པོར་བསྐྱེད༔}\\
\\
Під час сонячного та місячного затемнення, \\
Коли тремтить земля і гримить грім, \\
Під час сонцестояння та у Новий рік \\
Уявляйте себе у вигляді Самантабхадри. \\
\newpage
{\ti
ཀུན་གྱིས་ཐོས་པར་འདི་བརྗོད་ན༔ \\
ཁམས་གསུམ་སེམས་ཅན་ཐམས་ཅད་ཀྱང༔ \\
རྣལ་འབྱོར་དེ་ཡི་སྨོན་ལམ་གྱིས༔ \\
སྡུག་བསྔལ་རིམ་གྱིས་གྲོལ་ནས་ཀྱང༔ \\
མྱུར་བར་སངས་རྒྱས་ཐོབ་པར་འགྱུར༔}\\
\\
І якщо це сказано так, що всі вас чують, \\
Те, завдяки прагненню цього йогіну, \\
Усі істоти трьох світів \\
Поступово звільняться від страждань \\
І врешті-решт досягнуть просвітлення. \\
\\
{\ti
སྨོན་ལམ་རྒྱལ་པོ་འདི་དག་མཆོག་གི་གཙོ། །མཐའ་ཡས་འགྲོ་བ་ཀུན་ལ་ཕན་བྱེད་ཅིང་།\\
།ཀུན་ཏུ་བཟང་པོས་བརྒྱན་པའི་གཞུང་གྲུབ་སྟེ།\\
།ངན་སོང་རྒྱུད་རྣམས་མ་ལུས་སྟོངས་པར་ཤོག \\
།ཅེས་གསུངས་སོ༔ རྫོགས་པ་ཆེན་པོ་ཀུན་ཏུ་བཟང་པོ་དགོངས་པ་ཟང་ཐལ་དུ་བསྟན་པའི་རྒྱུད་ལས༔\\
སྨོན་ལམ་སྟོབས་པོ་ཆེ་བཏབ་པས༔\\
སེམས་ཅན་སངས་མི་རྒྱ་བའི་དབང་མེད་པར་བསྟན་པའི་ལེའུ་དགུ་པའོ༔ མངྒ་ལཾ།། །།}\\
\\
\scriptsize
Це прагнення, що має велику силу, завдяки якому
жодна з мислячих істот не зможе уникнути просвітлення,
вилучено з 9-го розділу тантри великої досконалості «Розчинена
реалізація Самантабхадри».\\
\\
На благо! На благо! На благо!\\
\\
Переклала Донла Санг'є Кхандро.
\normalsize
% HEVEA \end{document}
